\documentclass{article}
\usepackage{times}
\usepackage{fullpage}

\def\SVariable#1{\textit{#1}}
\def\SFunction#1{\textbf{#1()}}
\def\SArg#1{\textsl{#1}}

\begin{document}

In this homework, we will do several things.
We will parse HTML/XML output in R to
get data into R.
We will also see how  we can mimick a Web browser
and download pages, specifically dynamic pages,
from within a programming language like R.
And finally we will look at measures
for the commonality of Web pages.
Specifically, we will explore how 
Web pages that are returned for a Google
search refer to each other and/or common pages.


The first task is to go to the Google
web site using your browser and enter a query.
We want to retrieve the results and
construct a vector of the links for
the queries.
To do this, we need to parse the HTML
file returned by Google and extract the relevant
\verb+<a href="...">+ elements.
Save the result of the Google query 
to a file on your disk.
Then, use 
\SFunction{htmlTreeParse}
to parse the resulting file.
This will create a parse tree 
with all the elements.


Write a function that traverses the tree
and extracts all the links for the query.
Ignore the standard links to other Google
services, etc. and just return the ones
that are for the search.
You'll have to figure out how to identify those
generally.

Now, for each of the URLs returned by Google,
fetch the links in that page.
Use the hybrid parsing technique (i.e. via the 
\SArg{handlers} argument) to 
gather all these links.


Now we will try to setup a mechanism
to run the query directly from R
and avoid having to go to the browser
and save the file.
We want a function, call it \SFunction{google}, that
we can call with a query.
The function will send the query to the Google
Web server and get the result back.
To do this, you need to
use a socket connection and send the
request using the low-level HTTP protocol
that the Web browser and the Web server 
use to communicate.
Before we can send the query, we must establish
a communication channel between R and the Web server.
\begin{verbatim}
 con = socketConnection("www.google.com", 80, open = "w+", blocking = TRUE)
\end{verbatim}
We can then send and receive HTTP communications
between R and the Google server.

For the query "a simple query" that we would enter in the
Google text field for the search, the HTTP request is
\begin{verbatim}
GET /search?hl=en&ie=ISO-8859-1&oe=ISO-8859-1&q=a+simple+query&btnG=Google+Search
\end{verbatim}
When sending the query, we have to map certain characters to their
hexadecimal equivalent.
See \SFunction{htmlSubstituteCharacters} and
\SVariable{HTMLCharacterSet}.

To read the result back into R can be tricky.
We have to handle different possible communication mechanisms
from the Web server.  It can be as simple as using
\SFunction{readLines}.
If the server sends the response back in ``chunks'',
then we have to be more careful.
See \SFunction{readChunks}
and related functions.





Look at www.kartoo.com



\end{document}

