\documentclass[11pt]{article}
\usepackage{times}
\usepackage{fullpage}
\usepackage{hyperref}

\title{Relational Databases, Exploratory Data Analysis and Graphics}
\author{Duncan Temple Lang}
\begin{document}
\maketitle{}

There will be two parts to this assignment.  Neither of them should be
very hard.  The challenge is to interface to the database and, again,
think about how to map what you want into a form the computer can work
with.  This will involve interacting with both R and SQL and
determining where to do which computation.  And then you will want to
think of interesting and expressive ways to present the data.

Essentially, the task is to explore the data and pose and answer
questions you think might be interesting.  There is extra credit for
considering different problems and answering them using the data.  And
also, there is extra credit for thinking about different displays that
present the data in clear, meaningful and rich ways.


The databases are available on winnie.ucdavis.edu.  To access these
from R, you will need either the
\href{http://cran.r-project.org/src/contrib/RMySQL_0.5-3.tar.gz}{RMySQL}
package or the RODBC package.  On Windows, you can use either although
the RODBC package is preferred to simplify installation and minimize
problems in differences between the software components (i.e. MySQL
and RMySQL).  If you need an account on my machine, come by my office.
That will simplify installation procedures. However, you are
encouraged to install the RMySQL or RODBC package on your own machine.
It is a valuable experience.


\section{Network Traffic Data}
This data is taken from the
\href{http://www.ll.mit.edu/IST/ideval/data/1999/1999_data_index.html}{Lincoln
  Labs network intrusion detection experiment}.  They ``simulated'' a
running network for several weeks and gathered information about it
including all the TCP (Transmission Control Protocol) data that
identifies which machines and applications were communicating.  There
are many different interesting aspects to look at with TCP/IP data.
We will look at some relatively obvious aspects.  The focus here is to
deal with somewhat large volumes of data and to do some exploratory
analysis on the data by interacting with R and the database.

\section{Background: Transmission Control Protocol}
When an application like your mail agent or Web browser need to do
something with the outside world, they do so using TCP/IP.  When the
Web browser wants to download a page, say http://www.nytimes.com, it
connects to the host www.nytimes.com and goes directly to the ``Web
page'' department. Each department has its own door or \textit{port}.
This is port 80 for HTTP - the HyperText Transfer Protocol - used for
accessing Web pages.  The browser establishes a 2-way (bi-directional)
connection with the www.nytimes.com machine on this port.
And it uses this to send and receive information.  
The browser sends information such as 
\begin{verbatim}
 GET /index.html HTTP/1.1
\end{verbatim}
The Web server sends the resulting page back along this connection.

Each of these TCP connections is used to transfer the data that makes
up a dialog between the two applications.  Rather than sending files,
etc. TCP breaks the communications into smaller segments and sends
them in separate packets.  It is possible that a) some of these
packets may get lost on their travels from one application to the
other, and b) that they may arrive out of order.  Both of these are
problematic, but TCP corrects for them.

More information is available about the data on the course Web site.
And chapter 13 of Douglas Comer's book named Internetworking with
TCP/IP, Volume I: Principles, Protocols and Architecture (Prentice
Hall) provides more details about TCP and some of the characteristics
that might be interesting.


\section{Analysis}

Here are some questions to consider.  Do some exploratory data
analysis and present some thoughts about each.  You should provide
evidence including summary statistics, fitted models or graphical
displays.  Think about how to describe the characteristics
effectively, especially using graphics.  Think about displaying this
quantity of information on a single plot, or several plots.  Is there
too much? How can we present the different aspects clearly?  Consider
using color, different plotting characters (glyphs), line types,
marginal plots, conditional plots (e.g. lattice) and so on.  Hexagonal
binning is another option for dealing with large amounts of data.
These are meant as suggestions for those interested to investigate.

\begin{enumerate}

\item Look at the different types of connections (ftp, http, ssh,
  etc.)  over the different times of the day for each day.
  Are these characterized by ``bursty'' or sporadic behavior
  or are they reasonably stable?
  Is there a day-of-week effect? time-of-day effect?
  You might consider smoothing the data using something
  like the loess() function.

\item Are there any identifying characteristics of connections that 
  either last a long time or that involve a lot of data transfer?

\item Display the number of bytes sent in each connection.
    Does this vary by connection/application type.

 \item We are interested in packets that are lost and need to be
    re-transmitted.  Does this happen more at different times of the
    day?  When the overall ``load'' is high?

 \item Are certain applications more susceptible to idle times?
   Can we look at aggregate statistics for latency or idle times
   between packets or do we need to look at individual packets?
\end{enumerate}

These 5 questions are merely suggestions. Please (!) pose other
questions that you think might be interesting and discuss those in the
same manner as outlined for the 5 above.  There is extra credit for
making an effort in this direction and you can substitute some other
questions for the ones above.  You might think about looking at the
characteristics of the the connections on a machine when it has a high
and a light load, or similarly for the network as a whole being under
a heavy or light load.  You might look at different characteristics of
the TCP connections for different operating systems.  Lost packets
that needed to be retransmitted, duration of the connection, number of
bytes sent during the connection, idle time, etc. are characteristics
that can be explored either for the connection as a whole or for each
host involved in the connection.

Feel free to talk with me about details concerning TCP, or use the Web
to find information.  And you can look on the Web or in journals for
suggestions for plots, etc.  Please cite any works that you draw on.
TCP is an interesting topic as it is the infrastructure on which the
internet and networks in general are based.

Since there are over 1.3 million connection records and each tuple has
93 variables, dealing with this in R may be problematic.  You should
consider how to do the analysis by combining computations done in the
database and intermediate results transferred to R.

Your report should describe the data and discuss questions that you
explore, providing statistical support for any comments you make.
Also, explain how you did the computations, providing a high-level
view of the steps with details of the more complicated operations.
Reference any supporting functions you wrote and put these in an
appendix.  Make certain to provide numbers and captions for any tables
and figures.


\end{document}

\section{Baseball Data}

The purpose of thi
\begin{enumerate}
\item Check whether the data are correct.
 Ensure that the team payrolls match 
 the sum of the individual player salaries for
 each year.

\item Look at the relationship between 
  total payroll and the proportion allocated
  to the different roles
  such as specialized batters, pitchers, catchers and
  fielders.
  
\item Look at total payroll over time.
  Is there a relationship between going
  to the playoffs and total payroll?
  

\item What players played on a large number
  of teams and got paid more than the median?

\end{enumerate}

\end{document}
