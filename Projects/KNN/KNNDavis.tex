\documentclass{article}
\input{SMacros}
\usepackage{times}
\usepackage{fullpage}
\def\SFunctionRef#1{\textbf{#1}}
\def\Rpackage#1{\textit{#1}}
\def\SFunc#1{\textbf{#1()}}
\def\Svar#1{\textsl{#1}}

%\usepackage{html}

\newcommand{\Href}[2]{\htmladdnormallinkfoot{#2}{#1}}
\title{Cross Validation and the $k^{th}$-Nearest Neighbor 
Classifier of SPAM}
\date{DUE: Monday, February 28th, 2005 at Midnight}
\begin{document}
\maketitle

In this project, we focus on the problem of classifying email messages
as Spam or Ham.  We will consider two aspects of this problem: nearest
neighbor based classification methods and statistical
cross-validation.  The data we use are derived from the mail messages
from the Spam Assassin website
\texttt{http://spamassassin.apache.org/}.  The data frame is available
as an R data frame, with one record for each email and one column for
each derived variable. In fact there are two data frames, the one
called \Svar{derivedEmails} will be used to determine the
classification method and other called \Svar{\Svar{NewEmails}} will be
used to test the classifier developed on the \Svar{derivedEmails}
data.  The data frames are available on the class website in the data
directory:
\texttt{http://eeyore.ucdavis.edu/stat141/data/DerivedSpam.rda}.  A
description of the variables is included at the end of this document.

Our goal is to create a classifier to predict whether a new mail
message is spam or ham.  We will use what is called a $k^{th}$-nearest 
neighbor classifier, which finds the $k$ emails from a training
set that are closest to the new email and classifies the email as 
ham or spam according to whether the majority of the neighboring
emails are ham or spam.  That is, the $k$ nearest neighbors to the 
new email message vote for the message being ham or spam, according
to whether they are ham or spam. The majority wins. (What should
be done if there is a tie?)

To find the $k$ nearest neighbors to an email message, we need to
be able to compute the distance between any pair of emails.
How do we do that? 
This is where the variables come into play.
For each email in the training set
we find the distance between the value of the variables for the 
email in the training set and the new email.
To compute distances, we need a metric.
One well known metric is Euclidean distance, which may
be appropriate for continuous valued variables, but not
for logical variables. (Why?)
A binary metric may be more appropriate for  variables that
take on one of two possible values.  The R function
\SFunctionRef{dist} computes such distances for various metrics.

To determine the value of $k$, we want to compare the
performance of the $k^{th}$-nearest neighbor classifier for various values
of $k$.  At one extreme, we have nearest neighbor classification,
which takes $k$ to be $1$. 
That is, the new email is classified as spam or ham depending on
whether its nearest neighbor is spam or ham, respectively.
At the other extreme, we could use all the data in the training
set to classify the new email. In this case, the classification
would be the same for any email, i.e. it would be ham if the 
majority of the emails in the training set are ham.

To choose a value for $k$, we 
compare how well the method does when we know the truth,
i.e. when we know whether the new email is spam or ham.
We can compute the probability that a mistake is made.
There are two types of mistakes: a Type I error is when
an actual ham message is classified as spam, and a Type II
error occurs when a spam message is misclassified as ham.
We use cross-validation to determine the value of $k$ so that we
can control these errors. 


\section{Instructions}

\begin{enumerate}
\item Working with the \Svar{derivedEmails} data frame, consider the
  non-categorical data first.  Find the appropriate metric to use for
  these variables by themselves (i.e. without the categorical variables).
  The \SFunc{dist} function in R
  provides support for computing distances with six different metrics:
   euclidean, maximum, Manhattan, Canberra, binary
  and Minkowski.  The binary distance is not relevant here.
  The Minkowski metric requires that one specify a value for the power
  (p).  In order to determine which metric is good or ``best'', you
  can use k-nearest neighbors for different values of k such as k = 1,
  5, 10, 20.  If the same metric proves best for each of these, use
  that.  Alternatively, you can perform full-blown cross-validation
  and determine the best metric.


  The first issue you face is how to compute the distance between any
  two emails.  For continuous variables, such as the number of lines
  in the body of the email, the percentage of capital letters in the
  subject of the email, and the number of recipients, all but the
  binary metrics may be appropriate.  For example, for the  three variables
  \Svar{numLinesInBody}, \Svar{percentCapitals}, and \Svar{numRecipients}, we would compute
  the Euclidean distance between email message $i$ and email message $j$ as follows:

\begin{eqnarray*}
 [(numLinesInBody_i &-& numLinesInBody_j)^2 \\
   &+& (percentCapitals_i - percentCapitals_j)^2\\
     &+& (numRecipients_i - numRecipients_j)^2 ]^{1/2} 
\end{eqnarray*}

and the Manhattan distance between them is:
 
\begin{eqnarray*}
  |numLinesInBody_i &-& numLinesInBody_j|\\
    &+& |percentCapitals_i - percentCapitals_j|\\
      &+& |numRecipients_i - numRecipients_j|
\end{eqnarray*}

But the percentage of capitals in the subject
will always be between 0 and 1,
whereas the number of lines in the body of an
email can get quite large.
Should you standardize the continuous variables before
computing the distance between two records?
One way to do this is center the variable on
the median value and to scale it by the median
absolute deviation, i.e.
$$
\frac{(numLinesInBody - median(numLinesInBody))}
{mad(numLinesInBody)} .$$
Alternatively, the mean and mean absolute deviation could
be used to standardize a variable.
The Canberra distance includes a standardization.
 



\item Next, we will utilize the categorical data in our calculations.
  For logical variables, even if they are converted to 0-1 values, the
  Euclidean norm seems inappropriate to measure distance.  Instead,
  either the asymmetric binary or the symmetric binary metrics are
  more appropriate.  The \textit{binary} method for \SFunc{dist}
  implements the asymmetric distance.  For a given pair of records, it
  looks only at those variables for which at least one record has a 1,
  and then measures the proportion of these variables that are
  different for the records.  
%The rule for the asymetric distance is as follows:
%  compute the number of elements in the vectors for which at least one
%  of the two records has a non-zero value; count the number of these
%  elements for which the two records are different (i.e. one is a 1
%  and the other is a 0).  When neither record has any 1s, the distance
%  is 0.
An example may help. Suppose we look the three variables
\Svar{isRe}, \Svar{replyUnderline} and \Svar{multipartText}
and we have email records $i$ and $j$ as
\begin{verbatim}
derivedEmail[c(i, j), 
             c("isRe", "replyUnderline", "multipartText")]
      isRe replyUnderline multipartText
10    TRUE          FALSE         FALSE
1213 FALSE          FALSE         FALSE
\end{verbatim}
The distance is 1 (or 1/1) since there is only one
entry for which there is at least one 1 (i.e. TRUE).
And in this position (the first), the values are different so we count 1.
For two records of the form
\begin{verbatim}
   TRUE          TRUE         FALSE  TRUE
   FALSE         FALSE        FALSE  TRUE
\end{verbatim}
the distance would be 2/3.

Those interested might also explore the function \SFunc{daisy} in the
cluster package for R to use additional metrics such as the symmetric
binary distance.


Whichever metric you use, you then have to combine the distance values, $D$,
for the categorical and non-categorical variables (using the
non-categorical metric we determined earlier).  An obvious way to
combine them is
\[ D_{\textrm{non-categorical}}(i) + w D_{\textrm{categorical}}\]
Of course, we need to determine $w$, and we use cross-validation 
obtain values for both $w$ and $k$ in the k-nearest neighbors at the same
time.


\item Write an R function that uses $v$-fold cross-validation on 
the \Svar{derivedEmails} data frame to compare the predictive ability of the
nearest neighbor method for various values of $k$. 
It should take a distance matrix and $v$ as arguments
and return the $2\times 2$ classification table:
\begin{table}[htbp]
  \begin{center}
    \leavevmode
    \begin{tabular}{cc|cc|}
     & &  \multispan{2}{\hfil Predicted\hfil }\\
     & &  Ham  & Spam \\
      & Ham & Correct  & (Type I error) \\
     Truth & & &\\
      & Spam & (Type II error) & Correct \\
    \end{tabular}
  \end{center}
\end{table}
The Type I error rate is the rate at which ham messages are
mis-categorized as spam, and the Type II error rate is the rate at
which the spam is misclassified as ham. The logical vector that
contains the indicator for spam and a logical that contains an
indicator for whether the classification is correct or not can be used
to find these two types of errors.

R has several functions related to $k^{th}$ nearest neighbor in the
\Rpackage{class} package, but they use euclidean distance only.  Also,
some of the functions compute the $k^{th}$ nearest neighbor
classification for one $k$ at a time, but the distance between two
observations only needs to be computed once to figure out the $k^{th}$
nearest neighbor for values of $k$ from 1 to the size of the training
set.



\item Consider alternative metrics (i.e. values of $w$ or even choices
  of metric for the non-categorical variables), and use cross-validation to
  choose both the best $k$ and metric.  Plot the Type I and Type II
  error rates for the different values of $k$ and different metrics.
  Select a value for $k$ and a metric that defines a classifier with a
  ``suitable'' error rate.


Be sure to plot these two error rates (on the same plot) 
for various values of $k$ and various metrics.

In addition, you may consider exploring the behavior of choosing 
different values for $v$ in $v$-fold cross validation.


\item How well does your choice of $k$ and $w$ do in classifying the
  emails in the new set called \Svar{NewEmails}? Here you want to use the entire
  set of emails in the \Svar{derivedEmails} data frame as the training set
  and NewSet as an external validation set. (This is not
  cross-validation, but actual validation with new data.) Explain your
  findings using graphical and numerical analyses.  To do this, you
  might want to first examine the prediction made by the
  classification tree method in the next question, as it can give you
  an idea about which variables may be important.

\item Compare your optimal nearest neighbor procedure with the method
  of classification trees available in the \Rpackage{rpart} library.
  Use \Svar{derivedEmails} to settle on a tree (cross-validation is built
  into the rpart procedure) and then evaluate the tree's predictive
  abililty on NewSet.  Does the classification tree approach
  outperform the nearest neighbor approach?

  A classification tree is an intuitively simple classification
  method. Beginning at the root node of the tree, the data split or
  branch into two groups according to the value of a variable in the
  data frame. For example, the split may be according to the value of
  percentCapitals, where the data are divided into two groups
  according to whether the value of percentCapitals is above 0.11 or
  not.  Subsequent splits are made along the resulting two branches in
  a similar fashion, that is, each split considers the value of a
  single variable and subdivides the subset of data at that node into
  two subgroups.  The variables must all be either factors or numeric.
  The splits are made to make the resulting subgroups as similar as
  possible, i.e. to reduce the prediction error as much as possible.

  The \SFunctionRef{rpart} returns an object of class
  \SFunctionRef{rpart}. This means that many functions know how to
  handle this object. For example, if you use the \SFunctionRef{plot}
  with an rpart object, it will plot the subdivisions of the data as a
  tree.  Also, the \SFunctionRef{predict} function (see help on
  \SFunctionRef{predict.rpart} allows you to classify records using an
  rpart object.  To find out more about the \Rpackage{rpart} library,
  read the documentation at
  \texttt{http://cran.r-project.org/doc/packages/rpart.pdf}, or other
  documentation that you may find in a Google search.

\item Write up your findings in a 5-10 page paper, including plots and
  tables.  The focus is on the statistics, not the code. But provide
  an outline of your implementation of the algorithms you used.  The
  paper must be submitted in PDF.  In addition, submit your R code in
  a plain text file.

\end{enumerate}

NOTE: You are to work in groups of 2 for this project.
Only one report and set of code needs to be turned in.
Be sure to include in the report the name and SID number 
of each group member.


\begin{table}[htbp]
\begin{center}
{\Large Description of Variables }\vskip.3in
  \leavevmode
\begin{tabular}{llp{4in}}
isSpam & logical & whether mail is Spam (TRUE) or Ham (FALSE) \\
isRe                   &  logical  &  if the string Re: appears as the first word in the subject of the message \\
numLinesInBody         &  integer & a count of the number of lines in the body of the email message \\
bodyCharacterCount     &   integer & the number of characters in the body of the email message \\
replyUnderline         &  logical  & wheteher the Reply-To field in the header has an underline and numbers/letters \\
subjectExclamationCount&   integer & a count of the number of exclamation marks (!) in the subject of the message \\
subjectQuestCount      &  integer &  the number of question marks in the subject \\
numAttachments         &  integer &  the number of attachments in the message. \\
priority               &  logical & whether the message's header had an X-Priority or X-Msmail-Priority that was set to high \\
numRecipients          &  integer &  the number of recipients in the To, Cc fields \\
percentCapitals        &  numeric & the percentage of the characters in the body of the email that
are upper case (excluding blanks, numbers, and punctuation) \\
isInReplyTo            &  logical &  whether the header of the message has an In-Reply-To field. \\
sortedRecipients       &  logical & the recipient list is sorted by address \\
subjectPunctuationCheck&   logical &  whether the subject has punctuation or digits surrounded by characters,
e.g. V?agra and pay1ng, but not New!\\
hourSent               &  integer & the hour in the day the mail was sent (0 -- 23) \\
multipartText          &  logical &  whether the header states that the message is a multipart/text, i.e. with attachments. \\
containsImages         &  logical & whether the message contain images (in HTML)  \\
isPGPsigned            &  logical  & indicates whether the mail was digitally signed (e.g. using PGP or GPG) \\
percentHTMLTags        &   numeric &  the proportion of any HTML text in the message's body that is made up of HTML markup and not content. \\
subjectSpamWords       &  logical  &  whether the subject contains one of the following phrases:
viagra, pounds, free, weight, guarantee, millions, dollars,
credit, risk, prescription, generic, drug, money back, credit card.\\
percentSubjectBlanks   &  numeric  & the percentage of blanks in the subject \\
messageIdHasNoHostname &  logical  & whether the message id that uniquely identifies the message has no component identifying the machine from which it was set \\
fromNumericEnd         &  logical &  whether the user login in the From: field ends in numbers \\
isYelling              &  logical  &  whether the Subject of the mail is in capital letters \\
percentForwards        &  numeric & percent of the message's body  that is made up of content included from other messages  \\
isOriginalMessage      &  logical & body does not contain the phrase ``original message'' or something similar \\
isDear                 &  logical & whether the message body contains a form of the introduction Dear $\ldots$ \\
isWrote                &  logical &  whether the text includes a line indicating an included message as identified by the word wrote: in several different possible languages  \\
averageWordLength      &  numeric & the average length of the words in the body of the message \\
%XX
numDollarSigns         &   integer &  the number of dollar signs in the body of the message \\
\end{tabular}   
\end{center}
\end{table}


\end{document}
