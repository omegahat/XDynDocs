\documentclass[12pt]{article}

\usepackage{fullpage}
\usepackage{times}

\def\TA{Yu Chuan}
\def\SA{Sandy and Erica}
\def\InstructorMail{\texttt{nolan@stat.berkeley.edu}}
\def\TAMail{\texttt{yuchuan@stat.berkeley.edu}}

\def\SFunctionRef#1{\textbf{#1}}
\def\Rpackage#1{\textit{#1}}
\usepackage{hyperref}


\begin{document}

\parindent=0pt

\textbf{
Stat 133, Fall 04 \\
Homework 2: Handling Vectors, Arrays, Data Frames, and Lists\\
Due:  Monday, 20 Sep}

\medskip

For this homework, you are given a set of small tasks related to
manipulating lists and vectors and multi-dimensional arrays.
For each task, provide the R code to accomplish
the task, and give a one to two sentence explanation of your approach.
For example, if you are subsetting a vector, describe which of the
five methods of subsetting that you are using. 

\begin{enumerate}

\item Create a numeric vector of length 100 where the first twelve
elements are all 0 and the remaining cycle through the values 1, 2, 210. 

\item Create a numeric vector corresponding to the following population.
That is, there is one element in the vector for each unit in the population
and the value of the element corresponds to the value in the table below.

\begin{tabular}{rrrrrrrrr|r}
Value & 0 & 0.5 & 1 & 1.5& 2 & 4 & 7 & 30 & Total\\
Count & 83 & 15 & 23 & 17 & 12 & 2 & 5 & 1 & 158\\
\end{tabular}

\item Create a $3 \times 4$ matrix where the first row contains all 1s,
the second 2s, and the third 3s.

\item  Remove the last row from a data frame $x$ with an unknown number
of rows.

\item Set to zero all elements in a numeric vector $y$ that exceed 200.

\item Assign the first $n$ elements in the vector $y$ to the vector 
$tmp$.

\item Assign to the second element in list $z$ the character vector
$letters$.

\item Assign the second element in list $z$ to the variable 
$tmp$. Be sure to preserve the mode of the second element.

\item The data frame $chips$ contains variables $Transistors, MIPS,
Microns, Clock,$ and $speed$. Replace each value in $Clock$ by 1000
times its original value for only those cases that have a value of GHz
for $speed$.

\item  For list $w$ of length $m$ where each element is a numeric
vector (of varying lengths), find the maximum value for each of 
the $m$ vectors.

\item Create a three-dimensional array filled with uniformly distributed
random numbers. Find the sum of each ``page" of numbers. 
(See the \SFunctionRef{runif} function).

\item Simulate the distribution of the mean of a sample of size $n$
from a hypergeometric distribution. To do this, generate $N$ samples
of size $n$ from the hypergeometric distribution and compute the
$N$ means.
(See the \SFunctionRef{rhyper} function).

\end{enumerate}

\end{document}
