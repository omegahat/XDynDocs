\documentclass[10pt]{article}

\usepackage{fullpage}
\usepackage{times}
\def\SPAM{SPAM}

\def\TA{Bitao}
\def\InstructorMail{\texttt{duncan@wald.ucdavis.edu}}
\def\TAMail{\texttt{btliu@ucdavis.edu}}

\def\SFunctionRef#1{\textbf{#1}}
\def\Rpackage#1{\textit{#1}}
\usepackage{hyperref}

\date{}
\title{Stat 141, Winter 04 \\
    Project: Stage 1\\
    Reading \SPAM{} data \\
   \textbf{Due:  Wednesday, 28th, 5:00 pm}\hfill
}
\begin{document}
\maketitle

A collection of mail message data is available in the R package
\Rpackage{RSpamData}.  It is installed on the Linux machine already,
and it is available for download to other machines from the
\url{http://winnie.ucdavis.edu/stat141/Winter04/RPackages/RSpamData_1.0.tar.gz}{class
  Web site}.  The data were obtained from
\url{http://spamassassin.org/publiccorpus/}
which is a \SPAM{} filter that is widely used today.


There are five (5) directories in the dataset: three contain regular
mail (termed HAM), two contain \SPAM.  There are over 9,000 messages
in all.  Read the \url{http://spamassassin.org/publiccorpus/}{Spam
  Assassin Web page} to find out about the naming convention for the
files.  In the package, these five subdirectories are installed under
the \textbf{messages/} directory.

Your high-level tasks in this homework are to:
\begin{itemize}
\item  ``Read" the data into R. 
\item Explore the data and see if you can find any patterns
 to spam and regular mail that would allow us to
 identify either in some "automated" manner.
\end{itemize}

Each message is as it would appear in your mail folder.  It consists
of header information and the text of the message (see the sample
message below).  The text is just the regular text one would read and that
somebody wrote for the recipient.  In this format, attachments are
contained in the message and identified in very simple ways, typically
separated by a ``Boundary" identifier and ``Next part".

The header information is analogous to an envelope in physical
mail. It identifies the intended receiver and, for regular mail, the
identity of the sender.  It has a time and date stamp to say when it
was sent.  It has routing information about what machines it was
sent through on its way from the sender to the recipient.  There is
information about the subject matter, whether this is a fresh topic or
sent in reply to an existing mail message.  Each mail agent is
entitled to add its own information to the header.  Fields such as
mailing list information are often added.  Likewise, sometimes the
name of the program which was used to compose the mail is included in
the header.  The MIME (Multipurpose Internet Mail Extensions) type of
the message identifies the format of the body.
Each filed in the header is identified by its name followed by a colon
and the field information.

Note that the header and the body of a message are separated by an
empty line.


Before reading the data into R, what is a convenient way to determine
the proportion of \SPAM{} messages?


\section{Reading the Data into R}
How do we want the data to be arranged after we have read it in?

We want a collection of all the mail messages found in the five directories.
Each message should be broken into three parts: its header information, the body,
and an indicator whether it is spam or not.  The body can be further broken
into the regular text, included messages and attachments. For the
moment, we will not do that.

The header should be represented as a named vector, where each
header field constitutes an element of the vector.  The name should
give the name of the header field, and the element should contain the header
field value.
You will have to deal with fields in the header that
span multiple lines, e.g.

\begin{verbatim}
Received: from localhost (localhost [127.0.0.1])
        by phobos.labs.netnoteinc.com (Postfix) with ESMTP id E3D7B47C66
        for <zzzz@localhost>; Thu, 22 Aug 2002 09:54:39 -0400 (EDT)
\end{verbatim}

Note that all of these ``continuation" lines can be identified
by the fact that they are indented, i.e. start with white space.

We identify whether the message is spam or not from the directory in
which it is located.  The messages in the directories spam and spam\_2
are \SPAM, the remainder are not.


What sort of R data structure should you use to store the data?
Think about the nature of each ``record" and also what we will
want to do with it in the next stage of the homework.

When reading in the data, you will be both learning about R and 
understanding the details of the data.  It is often helpful to
explicitly raise errors when your code encounters something it can't
yet handle.  If you see the error, then that special case is contained
in the data and you will need to handle it.  If the error is not 
encountered then you may not have to add unnecessary complexity to the code.

To help you, I have put two functions (\SFunctionRef{readAllMessages}
and \SFunctionRef{readMessages}) into the  R package \Rpackage{RSpamData}.  The first function will do
the necessary looping over the directories containing the messages,
finding the top-level directory appropriately in a system independent
manner.  The function \SFunctionRef{readAllMessages} calls
\SFunctionRef{readMessages}, but the latter doesn't do anything at
present. It is just a template for you to fill in. It should contain the code to actually
read the messages in the different directories. The template establishes the
arguments with which the function can be called and serve as inputs to your
code.  There are help pages for each function.


\section{Exploring the Data}

Once you have the data, explore it to understand what we have and the
basic characteristics of the mail messages.  You might be interested
in some of the following features:
\begin{itemize}
\item 
   the distribution of the length of a message
\item
   the distribution of the number of attachments
  (can we identify attachments in the messages?)
\item
   information about who sends mail, who receives mail
\item
   the times that messages are sent
\item
  what mail programs (e.g. the "X-mailer" field in some headers)
  appear to be commonly used 
\end{itemize}
Some of these are not interesting,
and you should think about what other information
is in the data that might be worthwhile to investigate.

We will be using this data to try to classify new messages as \SPAM{}
or regular mail. So we are primarily interested in the differences in
the distributions of various variables for \SPAM{} and regular mail to
see if we can find some distinguishing characteristics.  How we can
see this? Do you think the differences are real? or just artifacts of
this particular dataset?   How might we tell?




\section{Useful R Commands}
The following are the names of some R functions that might be useful.
These are just hints. You should look at the help pages for these to
see what they do and follow the references to the other associated
functions given in the `See Also' sections of the help pages.

\begin{itemize}
\item Input/Output: \SFunctionRef{list.files},
\SFunctionRef{readLines},
\SFunctionRef{scan},
\SFunctionRef{list.files},
\SFunctionRef{system.file}.
\item String Manipulation:
\SFunctionRef{nchar},
\SFunctionRef{substring},
\SFunctionRef{strsplit},
\SFunctionRef{grep},
\SFunctionRef{gsub}.
\item Vector Operators:
\SFunctionRef{unlist}, \SFunctionRef{sapply} and \SFunctionRef{lapply}.
\item Error Handling:
 \SFunctionRef{stop}, \SFunctionRef{warning},
\SFunctionRef{try}.
\item Statistical Summaries:
\SFunctionRef{summary},
\SFunctionRef{table},
\SFunctionRef{plot},
\SFunctionRef{par},
\SFunctionRef{layout},
\SFunctionRef{hist},
\SFunctionRef{boxplot},
\SFunctionRef{lm}.

\end{itemize}


\section{To Submit}
Prepare a short report discussing both the processing of the data to
get it into the system and what you discovered when exploring the
data. You should: 
\begin{itemize}
\item
  describe what you did to read the data in,
\item
  include what challenges you encountered with data, 
\item
  present the 3 (or more) most interesting statistical features of the data --
  some of these features should relate to the differences, or lack
  thereof, between  regular mail and spam,
\item
  demonstrate and justify these features using
  numerical and graphical summaries.
\end{itemize}

Please create the report as a PDF or HTML file. To do this, you can
use Word, PageMaker, Mozilla composer, LaTeX, etc.  You can generate
figures/plots in different formats (e.g. JPEG, PDF, $\ldots$) in R
that should suit your needs.  Use the class bulletin board to ask
questions, discuss tools, etc. or ask \TA{} or me in class, office
hours, $\ldots$.

You should submit 
\begin{itemize}
\item the report
\item your code, arranged in files with the extension .R
\item any other auxiliary material such as plots, etc.
  that are not directly  contained in the plot
  (such as images and pages that serve as links from an HTML page).
\end{itemize}
Do this by creating an archive that contains all the files within a
single file.  You can use either ZIP or tar to create the archive.

Submit your work via email to \InstructorMail{} and \TAMail{}.

Make certain to also save a copy of what you submit.



\section{Sample Message}

\begin{verbatim}
Return-Path: whisper@oz.net
Delivery-Date: Fri Sep  6 20:53:36 2002
From: whisper@oz.net (David LeBlanc)
Date: Fri, 6 Sep 2002 12:53:36 -0700
Subject: [Spambayes] Deployment
In-Reply-To: <LNBBLJKPBEHFEDALKOLCIEJABCAB.tim.one@comcast.net>
Message-ID: <GCEDKONBLEFPPADDJCOECEHJENAA.whisper@oz.net>
                                                                                
                                                                                You missed the part that said that spam is kept in the "eThunk" and was
viewable by a simple viewer for final disposition?

Of course, with Outbloat, you could fire up PythonWin and stuff the spam
into the Junk Email folder... but then you loose the ability to retrain on
the user classified ham/spam.

David LeBlanc
Seattle, WA USA

> -----Original Message-----
> From: spambayes-bounces+whisper=oz.net@python.org
> [mailto:spambayes-bounces+whisper=oz.net@python.org]On Behalf Of Tim
> Peters
> Sent: Friday, September 06, 2002 12:24
> To: spambayes@python.org
> Subject: RE: [Spambayes] Deployment
>
>
> [Guido]
> > ...
> > - A program that acts both as a pop client and a pop server.  You
> >   configure it by telling it about your real pop servers.  You then
> >   point your mail reader to the pop server at localhost.  When it
> >   receives a connection, it connects to the remote pop servers, reads
> >   your mail, and gives you only the non-spam.
>
> FYI, I'll never trust such a scheme:  I have no tolerance for false
> positives, and indeed do nothing to try to block spam on any of my email
> accounts now for that reason.  Deliver all suspected spam to a Spam folder
> instead and I'd love it.
>
>
> _______________________________________________
> Spambayes mailing list
> Spambayes@python.org
> http://mail.python.org/mailman-21/listinfo/spambayes
\end{verbatim}

\end{document}

<title>Homework 1: Reading <SPAM/> data</title>

<!-- Use the code to catch errors in your understanding   -->








