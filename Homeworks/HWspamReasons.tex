\documentclass[12pt]{article}
\usepackage{fullpage}
\usepackage{times}

\def\TA{Yu Chuan}
\def\SA{Sandy and Erica}
\def\InstructorMail{\texttt{nolan@stat.berkeley.edu}}
\def\TAMail{\texttt{yuchuan@stat.berkeley.edu}}

\def\SFunctionRef#1{\textbf{#1}}
\def\Rpackage#1{\textit{#1}}
\usepackage{hyperref}


\begin{document}

\parindent=0pt
\parskip=15pt

\textbf{
Stat 133, Fall 04 \\
Homework 3: Text Manipulation: Creating Spam-related Variables  \\
Due:  Monday, 4 Oct}

\medskip

For this homework, you need to create three variables from the 
email messages. These messages are available in an R dataset in
the rda file, located at 
\\
http://www.stat.berkeley.edu/users/nolan/stat133/data/Emails.rda.

Descriptions of thrity variables appear below.
They are split into three groups: A, B, C.
You are to transform the email data into three of these variables,
one from each group. 
The list found at the end of the homework determines
which three you will write code to create. 
For at least one of the three your code must be in an R function.

Email the code you use to create these three variables, as 
plain text in the body of your email. In addition, turn in
a graphical comparison of spam and ham for each of these
three variables.  Discuss whether or not you think this 
variable will be useful in predicting if an email message
is ham or spam.


GROUP A:
\begin{enumerate}
\item The subject is "Re: something or other."
\item The number of lines in the body of the email.
\item The number of characters in the body of the email.
\item The Reply-To has an underline and numbers/letters.
\item The number of exclamation marks in the subject.
\item The number of question marks in the subject.
\item  The number of attachments.
\item The X-priority or X-Msmail-Priority set to high.
\item The number of recipients.
\end{enumerate}

GROUP B:
\begin{enumerate}
%\item The number of words in the body of the email.
\item[averageWordLength] The average length of words in the body.
% Some very large values because we get URIs, etc.
% Really want to parse and tokenize.
\item The Received time in the current time zone.
\item[fromNumericEnd]  The From: ends in numbers, e.g. 
\begin{verbatim}
david gezi <davidgezi12@hotmail.com>
\end{verbatim}
\item The subject is all capitals (excluding punctuation and numbers)
\item The body of the email contains another email. 
\item[subjectSpamWords] The subject contains one of the following words:
viagra, pounds, free, weight, guarantee, millions, dollars,
credit, risk, prescription, generic, drug, money back, credit card.
\item[messageIdHasNoHostname] The Message-Id has no hostname.
% 1 message has no Message-Id field.
\ite The message is HTML-only but contains no HTML tag.
\itemm[percentSubjectBlanks] The percentage of blanks in the subject.
\item The font color and background color are the same.
\end{enumerate}

GROUP C:
\begin{enumerate}
\item The email contains the recipient's email address. 
\item[sortedRecipients] The recipient list is sorted by address
\item[sibjectPunctuationCheck] The subject has punctuation or digits surrounded by characters,
e.g. V?agra and pay1ng, but not New!
\item  The difference between the Date and the Received date (be
careful with time differences.
\item The header states that the message is multipart,
but it is mostly text or html.
\item[containsImages] The email contains images.
\item[hasRedHTMLFont] The body has an HTML font tag with the color red.
\item The body contains Dear something, such as DEAR SIR, or Dear Madam
\item[] The Message-Id has a hostname that does not match the senders hostname,
but does match a host name at a relay point.
\item[hasClickLink] The body has an HTML link with text
that has the word "click" or "push" in it (upper or lower case).
\item The percentage of the characters in the body of the email that
are upper case (excluding blanks, numbers, and punctuation).
\end{enumerate}

The emails you will use for this assignment are in the list Emails, 
where each element of the list contains one email message.
Each email is itself a list consisting of three elements:
\begin{itemize}
\item The element named ``header" is a named character vector,
where each name corresponds to a key in the email header
and the value of the element corresponds to the text following the
: in the key:value of the header.
\item The element named ``body" is itself a list, the first element
of which is named "text" and contains the body of the email message.
This element is a character vector, with one string per line in the
email message.
A second element, if it exists, is named ``attachments." This
element is a list containing one element per attachment.
The individual attachment element is a list of two elements --
one containing information about the format of the attachment and the
other containing and the contents of the attachment.
\item The element named spam is a logical vector of length 1 
that indicates whether the message is spam (TRUE) or ham (FALSE).
\end{itemize}

To determine which three variable you are to write the code to create,
look up the last letter in your SCF login, i.e. if your login is 
s133bu then your assignment is \#8 in group A, \#1 in group B,
and \#4 in group C.

\begin{tabular}{lllc|lllc}
A & B & C & SCF login & A & B & C & SCF login\\
\hline
1 & 1 & 1 & a  & 2 & 2 & 2 & b\\
3 & 3 & 3 & c & 4 & 4 & 4 & d\\
5 & 5 & 5 & e & 6 & 6 & 6 & f\\
7 & 7 & 7 & g & 8 & 8 & 8 & h\\
9 & 9 & 9 & i & 9 & 10 & 10 & j\\
5 & 3 & 4 & k & 9 & 2 & 6 & l\\
8 & 2 & 7 & m & 7 & 4 & 8 & n\\
6 & 5 & 9 & o & 5 & 6 & 10 & p\\
4 & 7 & 3 & q & 3 & 8 & 1 & r\\
2 & 9 & 2 & s & 1 & 10 & 3 & t\\
8 & 1 & 4 & u & 7 & 2 & 5 & v\\
6 & 3 & 7 & w & 5 & 4 & 6 & x\\
1 & 7 & 2 & y & 4 & 5 & 9 & z \\
\end{tabular}

\end{document}
