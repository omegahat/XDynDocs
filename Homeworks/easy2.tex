\documentclass[12pt]{article}

\usepackage{fullpage}
\usepackage{times}
\def\SPAM{SPAM}

\def\TA{Gang}
\def\InstructorMail{\texttt{nolan@stat.berkeley.edu}}
\def\TAMail{\texttt{liang@stat.berkeley.edu}}

\def\SFunctionRef#1{\textbf{#1}}
\def\Rpackage#1{\textit{#1}}
\usepackage{hyperref}

\begin{document}

\noindent
\textbf{Stat 133, Spring 04 \\
Homework 2: Read One E-mail Message\\
Due:  Friday, 13 Feb}

\medskip

The \SPAM package can be found on the web at 
\url{http://winnie.ucdavis.edu/stat141/Winter04/RPackages/index.html}.
These data are from the Spam Assassin website,
\url{http://spamassassin.org/publiccorpus}.
The data and associated R functions have been incorporated into an R package
called \Rpackage{RSpamData}.

\medskip

Your tasks in this homework are to:
\begin{itemize}
\item  Download and install \Rpackage{RSpamData} into your SCF account or your home PC.

\item Use \SFunctionRef{readLines} to read in one message from the hard\_ham messages.
To determine which message you are to read, take the last two characters of your SCF login
and map them to the digits 00001 through 0078, where aa maps to 00001, ab maps to 00002,
... ba maps to 00027, bb maps to 00028, and so on. 
Your message will be the one that begins with this 5 digit number.

\item Split the message into two parts, one containing the header and one the body.

\item The header is made up of name:value pairs. Some of these span more than one line
in the header. 
Fold the multiple lines into one long line so that the header is a character vector, 
with each element corresponding to a header field.

%\item Represent the header as a named vector, where the name of the element is the header
%field and the element in the vector is the value of the field.

\end{itemize}

Use the class bulletin board to ask questions, discuss
tools, etc. or ask \TA{} or me in class, office hours, $\ldots$

You should submit via e-mail 
the R code that you used to process your e-mail message.
(See the R function \SFunctionRef{history} and \SFunctionRef{savehistory} 
for ways of accessing and saving the R commands you executed in your R session.)

Submit your work via email to \TAMail{}.
Make certain to also save a copy of what you submit.
You will be using this code in your first project.

\end{document}
