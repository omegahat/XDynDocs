\documentclass[12pt]{article}

\usepackage{fullpage}
\usepackage{times}
\def\SPAM{SPAM}

\def\TA{Yu Chuan}
\def\SA{Sandy and Erica}
\def\InstructorMail{\texttt{nolan@stat.berkeley.edu}}
\def\TAMail{\texttt{yuchuan@stat.berkeley.edu}}

\def\SFunctionRef#1{\textbf{#1}}
\def\Rpackage#1{\textit{#1}}
\usepackage{hyperref}

\begin{document}

\textbf{Stat 133, Fall 04 \\
Homework 1: Chip History\\
Due:  Monday, 13 Sep}

\medskip

A small set of information about the history of chip manufacturing 
appears on the web 
\url{http://www.stat.berkeley.edu/users/nolan/stat133/data/chip.txt}.
These data are from the HowStuffWorks website,
\url{http://computer.howstuffworks.com/microprocessor1.htm}.

Your tasks in this homework are to:
\begin{itemize}
\item  ``Read" the data into R using one of the R functions:
\SFunctionRef{readLines}, \SFunctionRef{read.table}, or \SFunctionRef{scan}.
Explain in a brief comment the reasoning behind your choice of function to use.

\item Explore the data visually for patterns over time, and for relationships
between variables (eg. transistors, clock speed, mips).
Consider using the R functions
\SFunctionRef{plot}, \SFunctionRef{hist}, \SFunctionRef{boxplot}, and 
\SFunctionRef{pairs}.
%In your analysis consider the log transformation, and ratios of variables.

%\item Describe in one or two paragraphs what you have found in
%two of your plots.
\end{itemize}

%Write your assignment up as a PDF or HTML file. 
%To do this, you can use Word, PageMaker, Mozilla composer, LaTeX, etc.
%You can generate figures/plots in different formats in R that should
%suit your needs.  

Use the class bulletin board to ask questions, discuss
tools, etc. or ask \TA{} or me in class, office hours, $\ldots$.

Save two of your best plots as a PDF file and the R code in plain text file. 
Submit your work via email attachments to \TAMail{}.
Make certain to also save a copy of what you submit.

%You should submit via e-mail 
%\begin{itemize}
%\item the written assignment
%\item any other auxillary material such as plots, etc.
%  that are not directly  contained in the written assignment
%  (See for example the R function \SFunctionRef{pdf} for how to create
%  plots in pdf).
%\end{itemize}


\end{document}
