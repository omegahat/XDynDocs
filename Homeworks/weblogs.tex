\documentclass[12pt]{article}
\def\SFunction#1{\textbf{#1}()}
\usepackage{times}
\usepackage{fullpage}
\usepackage{comment}

\title{Homework 2 \hfill Stat 141 Winter '05 \\
Due:  Wednesday, 2nd February, Midnight.  \\
\textit{Using R for Exploratory Data Analysis of
WebLog Data.}
}
\author{Duncan Temple Lang}
\begin{document}
\maketitle



In this homework, we will use 
some of the exploratory
data analysis facilities in
R to learn about the features
of the Web log data and become familiar
with R itself.
You should think about how we might do
some of these things in the shell
and how R and the shell tools are different.
It is important to be able to 
identify the appropriate tools for a task.


\begin{enumerate}

\item Read the data from the file \textsl{OmegahatLogData.txt}
into R.
The result should be a data frame with as many rows as there are
records in the log file. Programmatically verify this
by computing the two numbers from R.
There should be 6 variables, and you can name them 
IP, Date, Document, Referral, UserAgent and Bytes

The IP addresses are to be treated as regular strings.
We will want to be able to do arithmetic, ordering, etc.
of the Dates field, so we need to convert it to a suitable data structure.
POSIXct (see the help for DateTimeClasses in R) is
the best;  POSIXlt is another alternative.
Document, Referral and UserAgent are just character strings.
And the Bytes field should consist of  integers.


\item Find the most active robots, and the most active 
(i.e. number of hits)
regular or non-robot requesters.  In other words, identify
the records associated with robots and its complement
and look at the frequency table for hits on each set of records.


\item 
For the entire period,  what were the most active days?
And the most active hours? Is there a period in the day that is most active.
Use plots and/or summary statistics to find and illustrate
the answer.
You can also look at whether the results are the same
if we think of ``active'' as meaning
number of requests or the number of bytes transferred.


\begin{comment}
\item  What are the common Web browsers being used
when accessing this site?  Look at this by
browser and operating system combinations.
You want to look only at non-robot records.
\end{comment}


\item
Figure out how to get the data from
the raw Web log files (omegahat.log*) 
into the data frame.
There are two possible approaches:
\begin{enumerate}
\item write R code to read each record in the raw files and obtain
  the relevant fields; or
\item transform the data using UNIX/shell tools we used in Homework 1
to a file that we can easily read directly as a data frame in R.
\end{enumerate}
Chose one approach (or both for bonus points) and write
the code to do it.
Clearly, I did approach 2. If you chose that one, you need
to write a script similar to mine and use the code in question 1. 
It is interesting to write R code for the entire task
and compare it with the use of a shell script and reading
the rectangular table of data it into R.


\item Describe an interesting feature of the data
using plots or summary statistics to support
your reasoning.


\end{enumerate} 


Again, hand in a PDF file giving the answers
to the problems.  The answers should contain
the code along with annotations describing it for a human
to read,  output from R and any plots you created.


When creating plots, take time to consider the best
way to display the information.
Consider different plot types, transforming the data
(e.g. using logs),  putting data for different categories
(e.g. robots and non-robots)
on the same plot and using a legend,
ensuring the scales are appropriate and the same across related plots.

And take the opportunity to discuss alternative approaches,
solutions, algorithms, code segments, etc.  This is a wonderful
way to learn how things work, and when one is better than the other.
There are bonus points for discussing the different approaches.


\textbf{Useful Functions in R}\\
\textit{Input and Text Manipulation}
\SFunction{read.table}, \SFunction{read.csv},
\SFunction{count.fields},
\SFunction{readLines},
\SFunction{scan},
\SFunction{system},
\SFunction{nchar},
\SFunction{strsplit},
\SFunction{gsub},
\SFunction{grep}.

\textit{Dates}
\SFunction{strptime},
\SFunction{as.POSIXct}


\textit{Summary Tables}:
\SFunction{cut}
\SFunction{by},
\SFunction{summary},
\SFunction{table},
\SFunction{ftable},

\textit{Graphics}: 
\SFunction{plot,}
\SFunction{hist},
\SFunction{boxplot},
\SFunction{matplot},
\SFunction{points},
\SFunction{lines},
\SFunction{text},
\SFunction{legend},
\SFunction{par},
\SFunction{lattice},
\SFunction{xyplot}.


\textit{General}
\SFunction{sort},
\SFunction{rev},
\SFunction{as.numeric}.


\end{document}
