\documentclass[coloremph,lightbackground,landscape,17pt]{foils}
\usepackage{times}
\usepackage{color}
\usepackage{fixseminar}
\usepackage{graphics}
\usepackage{hyperref}
\usepackage[display]{texpower}
\usepackage{stepitem}

\def\SFunction#1{\textbf{#1()}}

\setlength{\voffset}{-1cm}
\addtolength{\textheight}{2cm}

\addtolength{\hoffset}{-1cm}
\addtolength{\textwidth}{2cm}

\begin{document}
\definecolor{gray}{rgb}{.6,.6,.6}
\definecolor{darkblue}{rgb}{0,0,.5}
\definecolor{darkgreen}{rgb}{0,.5,0}
\definecolor{cornsilk}{rgb}{1,.97255,.862745}
\definecolor{moccasin}{rgb}{1,.8941176,.70980392}
\definecolor{khaki}{rgb}{0.941176,0.901961,0.549020}
%\pagecolor{cornsilk}
\pagecolor{khaki}


We have seen vectors and seen a little bit of subsetting.  In this
lecture, we will talk about
\begin{itemize}
\item vectorized computations
\item the recyling rule
\item the 5 forms of subsetting
\item assigning to subsets
\item lists
\item data frames
\item functions
\end{itemize}

% \setboolean{display}{false}
% \let\hidestepcontents=\displaystepcontents

\newcommand{\newslide}[1]{\newpage\begin{center}%
\color{darkblue}{\large\bf #1}\end{center}}




Here we show that S works on vectors element-wise in many cases. The
idea is that if we operate on vectors of length 1, we get what we
might expect.  However, if we operate on vectors of length n,
S works on these element-wise.
\newslide{Vectorized computations}
\liststepwise*{
  \begin{stepitemize}
    \item 1 + 2
    \item c(1, 2, 3) + c(4, 5, 6)
    \item 1:10 + 11:20
    \item paste("A", "B", sep = ", ")
    \item paste(letters, letters, sep = ", ")

    \item sapply() function for vectors\\
          more later.
  \end{stepitemize}
}
In the first case of paste, we have two vectors of length
1 and we concatenate them. In the second case, we
have two vectors with 26 entries so this adds
a dimension. And paste puts the elements together
and returns a vector of length 26.


The recycling rule is important because it gives us a simple way to
think about how element-wise operations will work when we operate on
vectors with different lengths.  The basic rule is that we replicate
the smaller vector to have the same length as the larger one.  This
will generate a warning if the length of the smaller is not a divisor
of the larger one as we will have to drop some elements when we
replicate.

\newslide{Recycling Rule}
\liststepwise*{
  \begin{stepitemize}
    \item  c(1, 2) + 3
    \item  c(1, 2) + c(4, 5)
    \item  c(1, 2, 3) + c(4, 5)?
    \item  paste("Sample", 1:10, sep=".")?
   \end{stepitemize}
}
A simple way to think of this is \\
 x + rep(y, length = length(x))
where x is the longer vector.
 


We saw subsetting before, but we will go over it again
and show how the recyling rule works here too.
\newslide{Subsetting}
\liststepwise*{
  \begin{stepitemize}
    \item 5 types of subsetting
    \item including by position \\
      x[c(1, 2, 3)]
    \item excluding by position \\
      x[ - c(1, 2, 3)]
    \item including by name \\
       x[c("a", "b" , "c")]
    \item including by logical vector
     x[x $>$ 10] \\
     x[!is.na(x)]
    \item ?  \\ 
     (left until later)
  \end{stepitemize}
}

The subsetting allows us to easily and conveniently create new vectors
from original ones by including or excluding elements.  We are
guaranted that we will get a vector back of the same type as the
original vector.  For example, a subset of a character vector is a
character vector, a subset of a logical

\newslide{Assigning to Subsets}{
  \liststepwise*{
    \begin{stepitemize}
      \item x = 1:10
      \item x[c(2, 3, 5)] =  100
      \item x[c(2, 3, 5)] =  c(100, 200, 300)
      \item x[c(2, 3, 5)] =  c(100, 200)

      \item  All the other subsetting techniques work here too!
    \end{stepitemize}
}

%\end{document}

\newslide{Empty Subsetting}
\liststepwise*{
  \begin{stepitemize}
    \item empty index (i.e. x[ ])
    \item  y = 0
    \item  x[ ] = 0
    \item 2 dimensional vectors = arrays \\
            x[1:10, ] \\
            x[, c("Col 1", "Col 3")]
  \end{stepitemize}
}


\newslide{Matrices and Arrays}
\liststepwise*{
  \begin{stepitemize}
    \item matrix(values, nrow, ncol)
    \item matrix( NA, 10, 3)
    \item Stored as vector with dimensions \\
       homogeneous types!
    \item dim, ncol, nrow
    \item rownames, colnames, dimnames
    \item Subsetting \\
           x[, ]
    \item Generalize to k-dimensional arrays. 
  \end{stepitemize}
}

\newslide{Lists}
\liststepwise*{
  \begin{stepitemize}
    \item Container for different types of elements
    \item list(1, rnorm(10), c("ab", "cd"), list(x = sum, y = mean(rnorm(10))))
    \item recursive, i.e. lists within lists
    \item Same as vector \\
         but allows any S object as an entry
    \item support names
  \end{stepitemize}
}

\newslide{List Subsetting}
\liststepwise*{
  \begin{stepitemize}
   \item 5 subsetting techniques work.\\
          and for assignment
   \item x[1:2]
   \item Always returns a list
   \item How do we get elements? \\
          $[ [$ and $\$$
   \item x\$name
   \item x[ [1] ]       
  \end{stepitemize}
}


\newslide{Data Frames}
\liststepwise*{
  \begin{stepitemize}
    \item Can't use matrices for general data
    \item Need list of variables, each with own types
    \item Want to view as 2 dimensional \\
       e.g. for subsetting observations
    \item \textit{data frame}
  \end{stepitemize}
}


\newslide{Vectorized Computations}
\liststepwise*{
  \begin{stepitemize}
   \item Loop with apply functions

   \item Calls given function for each 
     "element".

   \item lapply, sapply for lists \& vectors  
   \item 
 Xs =  lapply(1:100, function(x) rnorm(100)) \\
 sapply(Xs, mean)
    
   \item apply for matrices \& arrays
    
  \end{stepitemize}
}


\newslide{Functions}
\liststepwise*{
  \begin{stepitemize}

    \item Call a function as     \\
      matrix(1:10, 5, 2) \\
      matrix(, 5, 2) \\
      matrix(1:10,  ncol = 2)

    \item Match argument by name or position

    \item First match exact names
    \item partial matching of names \\
        "s" matches "sd"
    \item match by position
    \item remainder in \texttt{$\ldots$}.

  \end{stepitemize}
}

\newslide{How Functions Work}
\liststepwise*{
  \begin{stepitemize}

    \item Create a special workspace for each call \\
       call frame or environment
    \item Assign the inputs (arguments) to the parameter names
    \item Evaluate each expression in the \textit{body} of the
      function.

    \item Assignments in these expressions go into this
      call frame.

    \item Where to find variables \\
        in this call frame

    \item if not there, look in the \textit{parent}  environment \\
       typically the session workspace

    \item then walk along search path \\
        in R, chain of environments.
     

    \item Remember each call (not function definition) gets its own
     call frame.
  \end{stepitemize}
}

\newslide{Lazy Evaluation}
\liststepwise*{
  \begin{stepitemize}
    \item Lazy evaluation of arguments \\
      evaluate argument expression only when value is needed.

    \item Examples of lazy evaluation \\
  ifelse(TRUE, 1, {cat("This isn't printed\\n"); 2})

   \item substring = \\
    function(x, start, last = nchar(x)) { \\
      ...
    }    
  \end{stepitemize}
}



\newslide{Defining Functions}
\liststepwise*{
  \begin{stepitemize}
    
  \item Use \texttt{function} keyword
  \item specify formal arguments 
  \item body is collection of S commands \\
    usually inside  { }
  \item function(x, y) { \\
    (x + y)/2
  }

  \item return value is either \\
    from explicit \texttt{return(value)}, or \\
    result of last expression evaluated
  
  \item Where's the name of the function?

  \item Functions are objects \\
     assign them to variables \\
      f = function(x, y) (x+y)/2

  \end{stepitemize}
}

\newslide{Formal Arguments}
\liststepwise*{
  \begin{stepitemize}
   \item given by name \\
      separate by commas (,) 

   \item can have a default value \\
     rnorm = function(n, mean = 0, sd = 1)

   \item Can determine if an argument was supplied or not \\
     if(missing(x))  x = 10 \\
     same as default argument, but more flexible (and obscure).
    
  \end{stepitemize}
}

\newslide{Unnamed formal arguments}
\liststepwise*{
  \begin{stepitemize}
    \item Some functions take unnamed arguments \\
       list, sum \\
       par, plot

    \item identified by \texttt{$\ldots$} in the parameter list

    \item 2 purposes \\
        arbitrary number of arguments \\
        pass to other functions
    
  \end{stepitemize}
}

\newslide{Managing Functions}
\liststepwise*{
  \begin{stepitemize}
     \item Write in a separate file
     \item Use \textbf{source()} to read file into S \\
         evaluates expressions including assignments
     \item check function 
     \item edit file 
     \item re-\textbf{source()}
     \item Eventually document and put in package!
  \end{stepitemize}
}

\newslide{Debugging Facilities}
\liststepwise*{
 \begin{stepitemize}
  \item   when you get an error, find the call stack \\
    \SFunction{traceback} 

  \item \texttt{data(mtcars) \\
         lm( mp ~ cyl, data = mtcars)}

  \item traceback() \\
7: eval(expr, envir, enclos) \\
6: eval(predvars, data, env) \\
5: model.frame.default(formula = mp ~ cyl, data = mtcars, drop.unused.levels = TRUE) \\
4: model.frame(formula = mp ~ cyl, data = mtcars, drop.unused.levels = TRUE)\\
3: eval(expr, envir, enclos)\\
2: eval(mf, parent.frame())\\
1: lm(mp ~ cyl, data = mtcars)\\
 \end{stepitemize}
}


\newslide{Interactive Debugging}
\liststepwise*{
 \begin{stepitemize}

  \item \SFunction{cat} and \SFunction{print}! \\
    modifies the code \&  very blunt.

  \item Stop the computation when there is an error
   and allow the user to explore the call stack
   and the local variables in each.

  \item \texttt{options(error = recover)}
  
  \item recover provides facilities for examining \\
     the local variables  - objects() \\
     the call stack -  where \\
     step through expressions - n
     exiting - empty command or q

 \end{stepitemize}
}

\newslide{Batch or Post-mortem debugging}
\liststepwise*{
 \begin{stepitemize}
  \item For batch mode (i.e. non-interactive session) \\
      Post-mortem debugging \\
      \texttt{options(error = dump.frames)}

  \item Follow error by \SFunction{debugger}
\end{stepitemize}
}


\newslide{Monitoring Functions}
\liststepwise*{
 \begin{stepitemize}
  \item \SFunction{browser}
  \item f = function(x = 5)  {  \\
          if(x == 1 || x == 0) \\
          \ \   return(1) \\
          browser()\\
          x * f(x-1) \\
        }
 \end{stepitemize}
}


\newslide{trace}
\liststepwise*{
 \begin{stepitemize}
  \item Rather than modifying the function \\
     \texttt{trace(f, browser, at = 2)}

  \item Can use at the beginning or end of the call \\
     \texttt{trace(f, quote(print(x)), print = FALSE)}

  
  \item Call \SFunction{untrace} to remove the current
      tracing action for the function.

 \end{stepitemize}
}


\newslide{Loops}
\liststepwise*{
  \begin{stepitemize}
    \item \texttt{for(i in obj) { \\
            expressions involving i \\
     }}

    \item \texttt{while(condition) { \\
            expressions 
           }}

    \item Condition must be a logical \textit{value} (not vector of
    length $>$ 1)

    \item Often use apply, lapply, sapply more effectively!

  \end{stepitemize}
}








\end{document}
