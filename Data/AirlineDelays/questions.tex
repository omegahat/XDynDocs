
\documentclass{article}

\begin{document}

\begin{enumerate}
\item How many flights were there in each year ?
Give explanations for any anomalies.
\begin{comment}
Incomplete data for 2008 because the year is not complete.
What about 1998?
Did some airlines go out of business?
come into existence?
\end{comment}

\item Look at the distribution of number of flights for different 
airlines?

\item Examine the distribution of delay times
by airport? and by airline?

\item What about international flights?

\item How fast do planes travel?
Look at distance and time.
Is there are an effect for plane types, 
length of flight, e.g. planes can make up  time
over longer flights, so departing late is not so bad?

\item Can you identify clusters of flights that are all late
within a particular period?
For example, weather or air traffic control issues
or reduced number of runways being used would
affect all 
Does the reason for delay help us identify such events?


\item Can we find information about individual planes?
If so, can we trace the cumulation of delays due to 
a plane arriving in one airport late, and causing
the next flight to be late, etc.
Where does this happen least - East or West coast or 
in the middle of America? and at what time of the
day? 
At night, planes stop except for red-eyes
and so there is a natural clearing point.

\item Is there a season effect to the times?
e.g. winds, 

\item Are airports at high altitudes more susceptible
to delays? of a particular type?

\item Explore the reasons for delays
by airline, airport, etc.

\item Are there more flights on particular days of the week?
in particular months?
 holidays? going back to college? vacations?
How can we tell if these are vacations - e.g. destination?
Can we find the ``always running'' flights and the extra flights
and see their origin and destination.

\item Fit Jim Albert's model to the data for each year?
what if you 
http://learnbayes.blogspot.com/2008/01/modeling-airline-on-time-arrival-rates.html

\item Some airports are hubs, serving as a connection center,
 and are centrally located.
 Other airports are just big because they serve lots of people.
 Use the census data to calculate the population served by
 different airports and explore the relationship
  between that and the number of flights.

\item Examine congestion and cumulative build up as delays start to
  occur in one airport and so cause delays in others
because flights cannot take off from other airports when
their destination is backed up and then subsequent flights are 
late.

\end{enumerate}

\end{document}
