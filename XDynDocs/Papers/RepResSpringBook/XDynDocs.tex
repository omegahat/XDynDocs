\documentclass{article}

\begin{document}

\section{The Document as a Notebook/Database}

What is the basic concept.

Projections, filtering, querying, updating (correcting function
locations)

That this approach is richer than Sweave.
Using either an R package, a zip archive (same as Office Open and Open
Office container format), or putting the data, etc. all in the one document.


Connection with literate programming, writing help pages for
functions, ...  Same markup vocabulary.  Same toolchain and
computational model for processing the documents (i.e. LaTeX and then
Rd is something entirely different).

Another vocabulary and toolchain to learn for \LaTeX{} users
but not for word processor users. But a much richer framework
that open up many new possibilities that we will be able to 
explore given these primitives. Interactive documents, drill-down,
alternative paths/branches,  semantic web.

\section{Authoring Formats: Word processor and Docbook}

\section{The XML Tools: XML, XSL, XPath, XInclude, XPointer}

\subsection{XML Schema}

\section{Tools for Authors}

\subsection{}
For users of Microsoft Word, we provide
a template (.dotm) file which contains numerous styles
corresponding to the different markup elements, e.g. R code,
R plot, R lattice plot, R expr, R class, R function, ...
One can also create derived styles based on these and other
``abstract'' styles.

The Word template file also contains a toolbar that facilitates
applying these styles.  There are also definitions for various key
bindings to allow the styles to be applied via keyboard manipulation
rather than point and click.

Programmers and users wishing to customize how R objects are rendered
or inserted into a document can provide methods for the
\Rfunc{RtoWordProcessingML} function.

An alternative to the approach we take for Word is to use an XSL style
sheet to transform the Word document into another Word document. We
think it is better to use R to manage this transformation as we have
more control to deal with multiple files and can leverage the ROOXML
package. But it is quite natural to put the rules to handle
code identified by certain styles within XSL.


\subsection{Emacs}
There is 
\section{Opportunities}
SVG

\section{Challenges}

Dealing with mathematical content. We can use MathML.

\section{Caching, Code dependencies}
We have taken a different approach to caching and avoiding
computation.  Firstly, we allow the author to explicitly put the
output they obtained into the document.  They can use this rather than
reperforming the computations.  This is the advantage of having markup
for arbitrarily many (i.e. more than two) types of chunks.

We also can work at the level of individual variables and expressions
to determine what expressions within code segments need to be
reevaluated when the value of an individual variable is changed.  This
uses reasonably detailed but conceptually straightforward code
analysis that is possible in a language such as R.

This code analysis and high-resolution sub-expression dependencies is
important for the interactive documents we can create from the dynamic
documents.

\section{Future Work}
More on alternatives and different projections.

\end{document}
