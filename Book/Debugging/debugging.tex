
\begin{abstract}
  We'll look at the general area of how we correct our code as we are
  developing it to overcome the problems and ``bugs''.
  We'll look at the tools that can help us within the R language, both
  debugging tools and also some testing tools.
  More importantly, we'll discuss the \textit{process} of debugging and the
  mind skills, rather than then specific computational commands, that
  one needs to develop to perform debugging efficiently and
  effectively.  In addition to fixing bugs, it is also important to be
  able to create test cases that find bugs or ``verify'' that there
  are no bugs. This testing is an important part of ensuring that 
  code is more general  than the one case we test it on.
  We'll then discuss how to make code ``better'' in terms of making it
  go faster, and identify some general principles and tools that can
  be used to help this. Again, we want to focus on how to think about
  these general problems and not just explain the syntax of the
  commands. One needs to be able to adapt the approach for different
  cases and be creative in considering considering various paths.

\end{abstract}


\section{The art of debugging}\label{sec:artOfDebugging}
Ideally, we would be sufficiently careful and clever to write code so
that it worked the first time.  And furthermore, we would ideally
never have to modify it to do new things.  However, we generally make
numerous mistakes even in a small segment of code consisting of just a
few commands.  And we regularly return to code we wrote previously to
adapt it to do something slightly different than we initially
planned. Once we start making these changes, we often break the
original code and have to go back and try to fix that feature while
generalizing the code.  So, generally, we spend a lot of time fixing
things rather than just writing and then executing the code. We
oscillate between changing the code and (re-)running it to check that it
works or gather information about what is wrong.  Generally, the
development of computer commands and software is almost always an
iterative process and at each iteration we gather data about what
aspects of the code we might want to focus on in the next iteration.
This is very much like data analysis.  We explore data through
different numerical and graphical summaries trying to get an
impression about how we might summarize or model it adequately, and
then we proceed to fit these ``models''.  We then look at many
different aspects of the results (e.g. residual plots) and return to
the modeling process.  Debugging is often similar where we run code
and compute intermediate and final results and verify that they meet
our expectations or find out where they don't.

It is almost absurd to think that somebody will write even five lines
of code without some simple mistake. And mistakes come in numerous
flavors. Being able to identify these errors rapidly and accurately
saves a lot of time and importantly, allows us to continue to focus on
the task at hand.  It is so easy to get distracted from this and let
programming become a mahor ordeal and time-sink. Practice and
experience really helps reduce this distraction. When we are doing
data analysis, we do not want to focus on the computations necessary
to see the aspect of the data in which we are immediately
interested. But bugs and errors in our commands distract us yet
further, furthering the focus onto the computations.  And this makes
us less productive and becomes a serious problem as deadlines
approach.  So it is very important that we learn, polish and maintain
our skills in the art of debugging.  The main skills in debugging are
reasoning about what could have given rise to particular outcomes or
events and connecting the output to the code.  It requires a lot of
conjecturing and reasoning to eliminate and prioritize the likely
causes of the issues.  The bugs are specific to the problem and the
class of common bugs are often specific to a language and indeed often
to a specific programmer. But there is a general approach to and set
of mental skills involved in debugging that transcend particular
languages and are intrinsic to all computer debugging. Indeed, they
are general problem solving skills. They involve diagnosing problems,
gathering data and evidence, connecting it to possible explanations,
testing the likelihood of the different explanations by both abstract
thought and also performing other tests to gain more information.

There are several general approaches or principles of debugging.
Also, each language has its own error messages, debugging tools and
also its own types of common bugs. We of course will focus on R.
Debugging in lower-level languages such as C/C++ or Java involves a
quite different interface. But the thought process is very
similar. But for languages such as C/C++ which do not take care of
memory management for the programmer have a rich and frustrating class
of additional bugs. Fortunately, in R, we don't encounter these.
However, we do often run into cases where code runs slowly because
objects are not collected when we think they should. While these
aren't bugs, they can make some computations infeasible or at least
very slow. So they can become non-fatal bugs and we have to look at
these with a different  mind-set and with different tools.


If we find that a program gives strange or incorrect results, we neded
to find out why. Before embarking on a potentially lengthy debugging
session, it is important to make certain that it is the code that is
wrong and not your expectation of what the right answer is. It is not
as unusual as one might think that you the human and code writer have
made a mistake in determining the expected result.  If you start
changing the code to try to ``fix'' it, you will in face be breaking
it.  So it is very valuable to have some cases that you have worked
through manually, i.e.  on paper or with a different piece of code,
that you can use to verify the results. You need to think of different
typical and boundary-case inputs and determine the correct results.
Then you can test the code on each of these and if the results agree,
you can be more confident that the code is correct.  However you still
have not verified this but merely got the response for a small set of
inputs.

So now that we are dealing with a case where the code is producing the
wrong answer, we need to find the earliest point in that code that
gives the ``wrong'' or unexpected intermediate result.  If we have
just changed a particlar segment of the code, that would be the first
place to look. However, when we have very llittle information about
the potential cause, such as when it is not our code or we have
changed lots of different pieces we might have to do what is called a
binary search or bisection.  Essentially we consider the code as two
pieces and place a check at the mid-point. If the results look correct
at that point, we surmise that the error occurs in the second half.
If the results are incorrect, then we look in the first half. And we
can repeat this step many times by splitting the particular part,
i.e. the first or second part and keep going until we find the
particular expression or line of code that seems to be giving the
error.  At any point, we determine if the results are erroneous by
looking at a subset or all of the local variables.

Bisection is a very generic style of debugging that makes use of no
contextual information such as the nature of the erroneous results.
For example, we might have a vector which as length $0$ rather than
the same length as the inputs. Or we might have \NA{} values when we
didn't expect any and that causes problems when doing further
subseting.  Such characteristics will typically give one more specific
locations of potential expressions that are causing the problems.
Obviously, if a variable \rvar{x} has the wrong value, we should first
look at expressions which assign a value to \rvar{x}. If there is
nothing obvious there, we need look at the values going into those
expressions and see if those intermediate values are correct.

Regardless of whether we have some hunches or contextual information
about the likely location of the problem, what we have discussed is
looking at the values of different variables and expressions.  So we
need to be able to do this while the function is running.  A very
common approach that people use because of its apparent ``simplicity''
is to add code that prints the values of various variables.  Then, as
the code is run, we see these values appear on the console and we can
review them to reason about what might be wrong. We are looking at
evidence and adjusting our hypotheses or conjectures, eliminating
some, giving less and more credence to others, and positing new
explanations.

Generally, adding code that simply prints variables is not a good
idea.  It is a very basic approach to debugging. It is simple and so
used by beginners and we all continue to use it for ``quick''
debugging.  In R, adding simple call to \rfunc{print} or \rfunc{cat}
is not usually a problem. However, it sometimes happens that the code
we add has a bug or an error in it itself.  For example, we may call
\rfunc{cat} with a complex object that \rfunc{cat} cannot handle, or
we may have a type in the expression.  But more importantly, printing
out numerous intermediate values can yield too much output that is
hard to digest and hides the important information. So the output of
interest should be chosen judiciously. A second problem with merely
print the intermediate results is tat we often want to explore the
object.  We want to do dynamic calculations on that object to explore
it in ways that are suggested by its contents as we examine it. In
other words, we want to investigate the object(s) in ways that will
only become apparent when we look at it.  Of course, we can add
additional commands to print out these further aspects of the
object(s) and then run the example agai.




Assign values so we can get to the same point without having
to redo all the calculations. Necessary in lengthy computations.





\begin{abstract}
 Here we look at how to make code more efficient, i.e. 
 run faster and/or use less memory. There are tools that allow us
 to do this empirically by finding the ``hotspots'' in the code.
\end{abstract}

 Parallel programming, 
  snow  
  random number generation issues

%%% Local Variables: 
%%% mode: latex
%%% TeX-master: t
%%% End: 




%%% Local Variables: 
%%% mode: latex
%%% TeX-master: t
%%% End: 
