What's this book about?  It is about trying to a) provide the reader
with the essential tools for working with data using R and other
technologies, b) expose the reader to the thought process of
approaching problems and mapping them into a series of computations,
c) to encourage readers to work with and explore raw data to answer
interesting quesionts and to focus less on the mechanics of fitting
models and applying statistical methodology for its own sake, and d)
to foster an appreciation for the availability of data that we can
access to understand social and scientific issues.  Essentially, just
as we want the public to be statistically literate and critical, we
also want students at all levels to be computationally literate so
that they can easily make use of this wonderful tool that has
radically altered science, society and even the way we learn and
communicate ideas.

We want to expose our readers to some modern statistical
methods that is a little more interesting than the classical
hypothesis testing. We don't want to focus
We want our students to be able 

Many statisticians started out studyig mathematics and gradually
drifted to statistics because of the potential to be involved in
analyzing data from many different scientific fields and applications.
However, statistics is taught and studied academically very much from
a mathematical perspective.  Much of this is historic. Certainly,
mathematics can be used to great effect to succinctly prove
characteristics of a particular methodology, and often mathematics can
illustrate a concept.  But we know that many people do not see things
mathematically or have to work very hard to understand and often have
an incomplete picture from the mathematical description of a
statistical concept.  Mathematics is not necessarily the one, natural
medium for describing statistical concepts.  It was the only one we
had for many years. But now the computer is an additional medium.  And
it is one that is much more active and exploratory than mathematics.
And it acts as a lab for us statisticians.  We can perform experiments
to explore statistical concepts, and create our own conjectures and
try to find proof for them and allows us to engage in experimental
statistics.  But most of all, the computer allows us to access vast
amounts of diverse sources of data which we can use to explore real
phenomena, verify common wisdom and debunk myths.  We can participate
in scientific exploration with other scientists, bringing our
statistical perspectives and insights. We don't have to wait to be
invited to participate in a problem (but of course it is best to work
with domain-specific scientists) before exploring some data.

Nowadays, scientists in many different fields are using statistics as
part of their regular workflow.  And they are using modern statistical
methods, some of which a statistics graduate student would not see in
any of her courses.  This suggests perhaps that our teaching may be
missing essential aspects, and focusing on traditional, fundamental
concepts at successive levels of mathematical complexity.  But one of
the aspects of our education that we sorely miss in statistics is
computing.  Partly because we focus on mathematical approaches to
understanding statistical concepts, we relegate computing to being a
skill of secondary importance. But this is also partially due to the
fact that many of us were educated before computers became ubiquitous
and so don't necessarily appreciate their growing importance and also
aren't as familiar with the fundamentals of computing as would be
desirable.  The curriculum has been slow to change to recognize the
vital role that computing plays in science and statistics.  For
statistics to continue to blossom and have relevance and for students
graduating with a statistics degree, it is essential that we embrace
computing, use it as a tool as we do mathematics and get on with what
we are focused on which is making sense of and interpreting data.  We
spend at least 10 years learning mathematics in school and college and
yet probably a tenth of that learning about computing.  And what we do
learn about computing is mainly basic and sufficient, e.g. choose the
File menu, and select Save As....  Understanding computing
fundamentals makes it quite easy to leverage new technologies as they
come along.  Having a superficial knowledge of ``skills'' or
``computing how-to-s'' does not give us this flexibility.


Many statistics classes teach methods that are general useful.  For
example, linear models, ANOVA, classification, and so on.  And these
are excellent things to know and understand. But what is vitally
important and often missing as a dominant component of courses is the
practice of analyzing data. This is what is interesting for so many of
us. We were drawn to statistics because of this opportunity to ``play
in other people's back garden'' (Tukey). What we want to do in this
book is to both teach you the basics or fundamentals of scientific
computing for working with data and also encourage you to be able and
want to go out and find data and answer questions that you find
interesting.  Not all the examples/case studies we present will
interest all people. And many of them are chosen because they are
manageable and relatively easy to understand without hours of
background reading.  But there are so many interesting problems and
questions and so much data available to explore them, be it simply
summarizing the existing data in interesting, illustrative ways or 
offering insights into the phenomena from which the data came based on 



%%% Local Variables: 
%%% mode: latex
%%% TeX-master: t
%%% End: 
