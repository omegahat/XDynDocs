
\section{Overview}
This chapter discusses some of the ways that we can make the results
of statistical analyses or the template for the analyses and methods
themselves accessible to people who are not familiar with the R
language.  This is a very important part of the statistical
collaboration or consulting process. In today's digital age and also
given the prevalence of data-related problems, using the Web or
dynamic electronic techniques to allow the large number of consumers
of statistical analyses to do the basics themselves by using crafted
templates is important as there are relatively few statisticians.
So we want to find ways to make our work available to scientists, etc.
in a way that puts them in control without them having to learn
a  new programming language and an awful lot of complex statistics.

There are several technologies that allow us to present 
analyses  templates that use R in a way that a researcher
can enter their own inputs and control the analysis at a high level.
One of them combines HTML forms in a Web browser and CGI scripts
on a Web server that sends the inputs from the user's browser
to a (remote) machine that performs the calculations and presents
the report, potentially with some additional files.

A second is that we make use of many people's familiarity with
Microsoft Excel on Windows. We let the scientist prepare the inputs in
Excel by giving them a template to edit and fill-in.  Then we connect
R with Excel and have R extract the relevant pieces and create the
results in an Excel file. The Excel template workbook can contain
interactive elements such as selection menus, radio buttons, spinboxes
to make it easier for the user and more constrained so that the inputs
are less ``free form''.

A third approach extends the idea of using a template GUI (Graphical
User Interface) in Excel or HTML to building a general GUI.
In HTML and Excel forms, the set of interactive controls is somewhat
limited and the layout is more constrained.  However, 
we can use one of several packages in R to program our own general
GUI entirely with R commands. We will illustrate some examples for 
doing this and focus on the RwxWidgets package.

We also look at creating ``dynamic documents'' using
XML and XSL and also Sweave.


%%% Local Variables: 
%%% mode: latex
%%% TeX-master: t
%%% End: 
