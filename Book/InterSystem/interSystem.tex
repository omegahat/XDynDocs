
\section{Connecting R with different Systems}

When we perform a complete analysis by fetching the data, cleaning it,
exploring it, ``modeling'' it and presenting the results in a highly
iterative cycle, we typically do not use just R.  Rather, we download
data from repositories often hosted on Web sites, we use the shell to
manipulate an archive of files and filter and transform some of the
content into more managable form. Often we will use richer programming
languages than the shell such as Perl or Python to do further
manipulation and exploration.  And then we get the data into R.  It is
very, very useful to know multiple programming languages so that you
can chose the most appropriate language for a particular task. Trying
to use R to do complex text manipulation, for example, can actually be
a waste of your time relative to doing it in Perl, if only you knew
enough Perl to get started. There are lots of issues about working
with two languages.  It is a context switch and one often ends up
using syntax from one language while programming in the other. And the
art of debugging, while general to programming, is context-specific
and one gets familiar with common errors and error messages.  Ideally,
we would be able to mix programming languages at a high-level and
learn just one.  In this chapter, we illustrate some approaches that
get us someways towards that goal.

We look at how we can use Perl code from within R. Specifically, we
will look accessing Bioinformatics data using Perl's BioPerl modules
without leaving the comfort of our familiar R session.  We also
illustrate how we can do this the other way around, i.e.  allow Perl
programmers to call R functions from within their Perl scripts.

R has several facilities for accessing data and code over the web,
e.g. \verb+source(url("http://www.omegahat.org/..."))+
Sometimes our Web requests or queries are more complex and we need to
send more inputs in the query such as criteria to identify the data of
interest, or provide our login and password or use a secure
connection. We'll see how we can do this with RCurl from within R
rather than 




%%% Local Variables: 
%%% mode: latex
%%% TeX-master: t
%%% End: 
