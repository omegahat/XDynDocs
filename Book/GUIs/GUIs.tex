A vital part of applied statistics is making the results and even the
tools available to scientists in a form that they can conveniently
use.  In this chapter, we will discuss some of the technologies that
we can use to make statistical analyses and results available to
people who do not know how to use the software in which the
calculations are done.  We'll outline several approaches which have
different advantages.  And, as usual, which one you might use depends
on different factors such as the nature of the computations, the
``users'' or audience, ease of deployment and installation, the time
available for the project, and how likely the project is to grow into
something more sophisticated.  But regardless of all these concerns,
the need for conveying the resultsis immense and unfortunately rarely
taught in statistics classes.  But an available, simple answer is
often better than an excellent, inaccessible answer.

In this chapter, we are interested in the idea of providing a
graphical user interface to a user so that she can input her data,
specify different parameters to control the calculations (e.g. the
value for $k$ in a k nearest neighbors model), and then see the
results as numerical and graphical output.  Of course, in R, we would
just provide them with a function and ask them to invoke that
function.  However, that assumes that they know how to install and
start R and find the function and then invoke it.  If you think about
whether you could do that with another language, e.g. ruby, then you
might have au nderstanding of why others would be averse to using your
function in R.  Unless she knows that the results are so important or
spectacular, there is a large disincentive to both to go through the
steps just to look.

A more natural reaction from a scientist might be to ask you to give
them a plot with the results.  And then after they have seen that and
think the approach is worthwhile, they ask you to do the same thing
for several datasets.  And, as is often the case, the choice of
particular tuning parameters is important and different across
datasets.  So would take you a non-trivial amount of time.  At this
point, you might feel that it is their responsibility and so you might
suggest that they learn R and install the relevant packages and your
code and fiddle with things themselves.  Often this is a viable
solution as the scientist might have a lab member do this for them.
However, that is not always feasible and teaching them how to use R
can, also, be time consuming for you.  So let's look at other
approaches.

We can assume that almost everyone you might be working with will have
used a Web browser. % blind people may not
And most will be familiar with using a search engine and specifying
the words to look for and getting the result back in a different page.
This is essentially the same thing that we want to do -- take inputs
and compute results and display them to the user.  So a Web browser is
often a good approach.  We need to be able to create controls for
specifying inputs and then we need to have these inputs be sent to a
Web server and processed by R to do the computations, and finally
generate the results as an HTML page.  This involves a client-server
style of computing in which the web browser is the client and we have
to have a Web server and R to act as the server.

The Web form and server approach is very useful as the user does not
have to install anything on her machine.  She just points the web
browser at a URL and fills in the form and clicks a button.
Unfortunately, that means that you have to setup the Web server. But
that may still save your time overall.  But a more problematic issue
with this approach is that the HTML forms are very limited.  We are
used to have sliders, spinboxes, data grids in spreadsheets and
hierarchical trees to select files within folders, yet the controls we
can have in HTML are much more limited and primitive.  We can have
radio buttons, check boxes, text areas and basic buttons for
submitting or clearing/resetting the form.  We have to use JavaScript
to do more interesting things with forms, and even then we still are
stuck with the basic controls.  If we want to build a richer GUI for
the users, we need a richer toolkit.  How are applications like MS
Word, Firefox, Thunderbird, OpenOffice, Gnumeric, etc. created? They
use one or more GUI toolkits to create the displays.  And there are at
least three such toolkits available to us in R which allow us to
create professional quality GUIs using only R code.

Of course, creating a GUI is not entirely trivial, especially from a
design perspective. And as it becomes more complex, users will have to
become familiar with the layout of the controls.  Some users on a
Windows machine might ask why can't you provide the functionality in
Excel. Of course, there are many reasons you might present such as a
desire to get the correct answers, or produce plots that aren't
egregiously offensive.  But there is a real issue here.  Many users
are very comfortable with Excel and they may want to bring their data
into a spreadsheet and transform and manipulate it before handing it
to our function(s).  Why should they have to save it to another file
and then load this into another GUI. Isn't there someway that the two
GUIs can be integrated?  And of course there is, and in fact, we can
actually build the GUI in Excel itself and still have R respond to the
user actions to update the results.

So we will see three or four different approaches:
\begin{enumerate}
\item HTML forms and an CGI script on a Web server,
\item an problem-specific GUI running within an R session,
\item GUI within Excel connected to R to handle the actions and computations using DCOM.
\item Google Earth (or GIS) to display points or plots on a map,
  potentially over time.
\end{enumerate}
Within each of these, there are lots of details
and different techinques that we can use.
And of course, there are many other languages we could use.


% Need to pick a case study and work through it for each of these different
% approaches.


\section{Web Servers}
mod_R
AJAX 

Web Services - R to  Java to TOMCAT to Globus

%%% Local Variables: 
%%% mode: latex
%%% TeX-master: t
%%% End: 
