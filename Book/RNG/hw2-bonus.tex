\documentclass[11pt]{article}
\usepackage{times}
\usepackage{comment}
\usepackage{fullpage}
\def\SFunction#1{\textbf{#1()}}

\title{
 Homework 2 Additional Questions \hfil Stat 141, Winter '06   \\
}
\begin{document}
\maketitle

These are  not required questions but are for those
who are interested to explore.

\begin{itemize}
  \item 

  Use the acceptance/rejection method to sample from a Gamma$(2, 3)$
  density.  If you are feeling adventurous, you can write code for
  general shape and rate/scale parameters, with scale $\ge 1$.  While
  one can do it generally, you can do experiments or write the code
  with a particular (non-trivial) $(\alpha, beta)$ pair.  But you
  should try it for different pairs then.
  
  One could use the uniform density to majorize the target
  distribution. Determine where the mode lies and what is the density
  at that value. That tells you what value of $c$ is needed to find $c
  g(x) \ge f(x), \forall x \in [0, \infty)$.

  To try to reduce the number of rejections, you can find a better 
  majorizing function, $g(x)$.
  You can try the double-exponential, defined as 
$$
  g_X(x) = \frac{\theta}{2}e^{-\theta \vert x - \eta \vert}
$$  
   choosing $\eta$ as the mode of $f_X(x)$.

   Use the computer to generate a sample from the Gamma$(2, 3)$ using
   these two majorizing densities. Ensure that the results are close to the
   Gamma$(2, 3)$ density by looking at Q-Q plots, or overlaying the Gamma
   density on a histogram of the sample values, etc.  

   Note that there are two approaches to sampling from a
   double-exponential. One is to use the inverse CDF method and the
   other is to work from \SFunction{rexp} and randomly
   decide which ones to make negative.

   Do both and determine which is more efficient. Is this
   efficiency a function of the parameters of the Gamma distribution?
   Can one calculate the (expected) rejection rate mathematically?

  \item 
   Consider a Beta$(2, 3)$ random  variable.
   It is defined for $x \in [0, 1]$.
   Use the uniform majorizing function
   to sample from this density.
   Also, consider the Epanechnikov
   density.
   This is defined as 
$$
  g_U(x) = \frac{3}{4}(1 - u^2), \qquad u \in [-1, 1]
$$ 
   If you scale and shift the range to $[0, 1]$, you end 
  up with a Beta$(2, 2)$ density which has the ``special''
  form
$$
  g_X(x) = 6x(1-x), \qquad x \in [0, 1]
$$
 Find a value of $c$ such that $c g_X(x) \ge f_X(x)$
 for all $x$ where $f_X(x)$ is the density of the Beta$(2, 3)$.
 You can use R to get an approximate value for $c$,
 by plotting the curves and/or evaluating the functions
 at many different values along the unit interval.
 
 Having decided on a value for $c$ for each majorizing function, plot
 the sampling regions for the two majorizing functions, i.e. $c g(x)$.
 Can you tell which one will be have a higher acceptance rate?

 For the Epanechnikov, you can calculate the CDF for this and then use
 its inverse to generate samples from $g_X(x)$.  

\end{itemize}

\end{document}
