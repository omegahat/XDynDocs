\documentclass[11pt,timesroman]{article}
\begin{document}
\baselineskip=18pt

\Goal{Make it easier to read a DBMS book after reading this material
and to understand how to use a database from within a statistical system
and  when to understand when it is appropriate to use an RDBMS to
manage data.
}

\section{Introduction}
Introduce different scenarios  as to how we come to use a database
\begin{itemize}
\item in industry, data collected from manufacturing process in
  databases and interested in the production process and improving,
  e.g., yield.
\item a clinical trial where data is gathered on observational units 
  for a variety of different purposes
\item information is gathered for tracking inventory and sales in
  Wal-Mart.  Different groups decide to ``mine'' it for 
 relationships to see if they can improve the Supply Chain Network
  (SCN), marketing strategies, etc.

\item you are starting a study   with different types of data (images,
 numbers, files, etc.) and a
 large quantity of it (e.g. from a collection source such as a
 computer network).  Rather than using some ad hoc solution to manage
 the data without knowing precisely how you will use it, you choose to 
 keep your options open and to use a general database system to manage
 the data.  S-Net is an example.
\end{itemize}


Cover topics such as 
\begin{itemize}
\item imposed on users because of corporate/instutional approaches to
  gathering and managing data 
\item meta-data
\item synchronization
\item client-server computing
\item security
\item performance (specialized)
\item  connections to data frames and statistical data ``models''
\item live data
\end{itemize}

\section{Basic Model - for statisticians forced to access a database}
\subsection{Relational Components - the Table}
 The basic unit is a table.  \\
 Query gives us back a table. \\
 algebra means we are always dealing with a table. \\
 Deal first exclusively with a single table.

 What is different from a data frame in S. 
 \begin{itemize}
 \item live/transactional data that means we get updated results if we 
   re-run the same query.  This is different from R's functional
   programming model.
\item uniqueness of rownames, etc.?
\item meta-information, schema and type information and checking.
\item NULLs and NAs
 \end{itemize}

\subsubsection{Queries and Query Language - SQL}
 Again, limit to a single table for now.

 Notion: Subsetting is the thing we do most of. That is one of the
 main reasons why we use S.

 Subsetting by records and columns/variables. - restrict and projection

 Translate the technology concepts for the audience and illustrate how
 to think about and exploit RDBMS from a statistical perspective. \\
   i.e. Relate to S.


\subsubsection{Data Types}
 Basic data types 
 
 Handling dates and times.

\subsubsection{Functions - Record and Aggregate}


\subsection{Multiple Tables and the Relational Model}

\subsubsection{Why - Motivation}
 Redundancy elimination, etc.

\subsubsection{Keys} 
  The relational logic

\subsubsection{Join}

\subsubsection{Examples}
 In both R with multiple data frames and then with tables.

\section{Accessing RDBMS from R}
Concepts of the model - RSDBI \\
  RODBC, RMySQL, RPgSQL 

Basics of the model
\begin{itemize}
\item driver
\item connection
\item query \\
       basic and dynamic content (e.g. from variables within R)
\item result set \\ 
       fetching in blocks

\end{itemize}

\subsection{Accessing RDBMS from other Systems}
Commonalities of the model.
\subsubsection{Perl/Python}
\subsubsection{Java}
JDBC
\subsubsection{Windows/Excel}
ADO


\section{SQL for Smart Statisticians}
Cookbook for tips on how to get results from a database
in ways that may be challenging or more useful for statistical
work.

Advanced issues.


\section{Understanding how a Database works}

\section{Alternatives to Databases}
Flat files, file systems, XML, Object databases, etc.

\section{Managing and Designing your own Database}
\subsection{When to use a RDBMS}
 When to use ad hoc/application/project-specific solutions and
when to move to a database.  Rules of thumb.

\subsection{How to setup a database server}
The server software, not designing the database schema themselves.

\subsubsection{Basic Commands}
CREATE DATABASE
CREATE TABLE
\subsubsection{Access, Privileges, Security}

\subsection{Designing Schema}

\subsection{Consistency}
\subsubsection{Handling transactions and elimation of records}

\subsubsection{Data Collection and Triggers}





\end{document} 
