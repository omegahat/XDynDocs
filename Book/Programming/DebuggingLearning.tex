If one is fortunate to have an (relative) expert 
in a subject, it is very tempting to ask that person 
for answers to all of your problems about that topic.
Of course, it saves time and it doesn't  
take that person long. So you
are more productive and the other person
gets to feel good that she knows something and
is being helpful. 
It is a win-win situation. Right! Of course
we all know that it consumes time from
the other person. She has to switch gear
and understand the general aspects
of your problem and then do the debugging.
That takes time and energy.   She has also
had to spend a great deal of time learning
the material and thinking about it in different
ways so that she has an comprehensive
understanding of it.  This takes time and 
effort that could be spent on other, more
direct results.  Yet it is a vital skill.
So it is a little unfair to constantly 
rely on others and to expect them to have
put in the time to have learned something
in depth so that you don't have to.
Of course, it is fine if everyone
specializes and we share our expertize in
freely in all directions and give as much as we take.
Computing does have a tendency to be used by many
fields and be in constant demand,
while high energy physics expertize is less 
ubiquitously needed on a day-to-day basis.

Let's also look at  things from the perspective
of the non-expert.
By deferring to others and allowing them to 
solve the problem, that person can move on
to the next step.
However, they have lost a valuable opportunity to
learn more.  And learning more will probably
be useful in the future when dealing with
the same problem in a marginally different context
or dealing with entirely different problems.
It is the process of learning about a
specific that allows us to learn general things.
Eliminating or even taking wrong paths 
in exploring issues
means that a single topic teaches us many things.
We need to be less short-sighted and focussed on
immediate results in the form of solving the
task directly ahead of us. Instead, especially as students
and researchers, amortize research over the longer
term. Obviously there will be deadlines that must
be met and constrain us at different times.
However, if they are persistent and we don't
feel like we have time to do anything else
but meet the next deadline, perhaps we need to 
look at the nature and quality of the work.
New and hard ideas take thought and time,
not merely activity.



Having returned to academia and having students come to ask me
questions about their R code has been very illuminating.  In Bell
Labs, we were fortunate to have highly skilled and motivated
researchers who all had a long history of working with data and doing
their own computing.  They have strong background in the S language
since it was developed there.  In universities, on the other hand, the
students typically do not have the same experience either in S or of
general problem solving.  The former is a specific skill; the latter
is a vital skill for scientists and tends to help greatly with the
former.  How we teach general problem solving is a very hard topic.
We can however narrow our focus to how to solve problems in S.  There
are two aspects to this.  One is to understand how to use the language
(the syntax) and the idioms and style to create good programs.  This
is an art, but a less individualistic art than many other forms.  The
second aspect is how to recover from errors that we make in the code.
To do this, we first need to identify the error, pin-point it, and
then determine how to fix it.


\section{Know what to expect}
When something doesn't work
because it either fails or gives you an incorrect
answer, determine what should have happened.
Think about the computations and
figure out the different steps so that
you have a rough model of what was being
done at step.
Then go back and look at any diagnostic
messages that were emitted during the computations
and compare those with what you expected.
If anything seems strange, investigate that.
This is an iterative process as your expectations
 might be wrong. So something that seems strange may not
be the cause of the problem, but investigating
it will help you to get a better understanding of
what is going.  It helps understand
both the error as well as the way you are thinking
about what you expected to happen.
You can modify both.


Consider the following example.  We create a simple package in R.  We
aren't going to require that it pass the more strict checking
facilities for packages, but instead just want to be able to install
it.

We arrange the files in the appropriate structure for an R package.
That is, we put the R source files in the directory \dir{R}.  Next, we
create the DESCRIPTION file.  The format of the DESCRIPTION file is
described in the \cite{WritingRExtensions} manual. Since it is
extended over time, we should keep up with the additional entries that
it supports or even requires by either rereading the manual or
checking the NEWS file providing with each new release of R.  Reading
about the contents of the file in the abstract without any particular
need to use them often means we don't remember them.  So a natural
inclination given that we are focussed on using the code as package
rather than exploring the details of the R package DESCRIPTION file is
to find one that you created previously and copy it.  Suppose we do
this.  We copy it from another package for which we have the source.
Just to be concrete, suppose we take the one from the XML package and
copy it to our package.
Then the phone rings and we are asked to do something urgently.
So we move to another virtual desktop and start on that.
After a half hour, we finish with that new task and return to our
package. 
We set about installing it. 
\begin{verbatim}
R CMD INSTALL myPackage
\end{verbatim}
Seeing that there were no errors, we
can assume everything went okay and 
now load the package.
So we start R
and call
\begin{verbatim}
library(myPackage)
\end{verbatim}
The result is an error telling us that there is no such package.  Mmm.
After a few moments of blaming the stupid software for getting it
wrong and not having paid attention to what you just did, questioning
the bright origins of the developers, and so on, you resign yourself
to the idea that while it is not your fault, it is your problem.
Where to start.
As Julie Andrews sang, ``the beginning is a very good place to start''.
Let's break down what we did into the high-level steps:
\begin{itemize}
\item created the package,
\item installed the package,
\item load the package in R.
\end{itemize}
Now the error could be anywhere.
In some ways, it would be simplest if it
was near where the error was.
R couldn't find the package ``myPackage''.
Did we spell it correctly?
Check the command
\begin{verbatim}
library(myPackage)
\end{verbatim}
against the name of the directory 
containing the package source.
These seem to be the same, but make certain this is the case.
If the package name is complex, we can write some R code to make
certain
they are the same:
\begin{verbatim}
> x = "myPackage"
> library(x, character.only = TRUE)
> any(list.files() == x)
\end{verbatim}
If the answer to the last command is 
FALSE, then there is typo somewhere.

Let's try to think about other explanations.  What if we are
installing the package into one place but R is looking for it in
another place.  When R installs packages, it installs them into either
a local place you specify or into the system area where R is
installed.  In many installations, regular users don't have
permissions to add packages to the system area.  Therefore, you have
to be adding them to a ``local'' area.  You can do this via the -l
flag for R CMD INSTALL, or you can specify it with the environment
variable R_LIBS.  Similarly, when R loads a library, it looks in
different places.  You can explicitly tell it where to look, or it can
use the value the R_LIBS environment variable, and lastly it will look
in its own installation or system area.
Now we can see a potential for a problem. Perhaps we installed
it in one place using one setting for  R_LIBS
and another when we tried to load it.
So we need to check this.
We check the environment variable in the shell
in which we installed the package using the
shell command
\begin{verbatim}
[shell] echo $R_LIBS
\end{verbatim}
and in R where we load the library
\begin{verbatim}
R> Sys.getenv("R_LIBS")
\end{verbatim}
Are they the same?
What if we tell R to look for the package in
the directory given by the first element of
R_LIBS from the one in the shell
\begin{verbatim}
R> library(myPackage,  lib.loc= "/whever/my/Rpackages/go")
\end{verbatim}
Okay, so it wasn't that. We could have known that 
since we were using the same shell
for installing and running R.
However, we might have been running the two commands in seperate
windows or shells.
And also, we could have manually set the R_LIBS variable
and have it explicitly reset in our shell startup scripts.
So it is always worthwhile to eliminate the possibilities.


In our case, the loading error is only a manifestation of the problem.
We have to dig more deeply.  Let's return to the first step.  Check
that the package is correctly structured.  It has the R directory and
there are (non-empty) files there.  The DESCRIPTION file is present.
Everything looks okay here.  So let's move to part of the process that
gave us some diagnostic output rather than looking at the package
which is something we created and believe we created correctly but
appears to have a problem and which is not actively telling us
anything.  The installation step produced the following output.  No
errors, so what could be the problem.  It was a source package, its
name is XML, it has no help files....  Wait a second - its name is
XML!  Oh yeah, I copied the DESCRIPTION file from the XML package and
I never did get to change its contents.
So let's do that. 

But before we do, let's think for a moment to determine if that is the
likely cause of the problem.  In other words, what are the
consequences of having a DESCRIPTION file from another package in
ours.  Well, we can try it and see if it works before thinking about
why or why not.  But, again, there is something to be learned by
thinking about this rather than actively trying things
in a haphazard or non-thinking manner.
The name of the package in the DESCRIPTION file is wrongly
given as XML and that is what was being installed.
So we have learned that the R installation procedure
takes the name of the package from the DESCRIPTION file
and not from the directory. Store that one away.
How would we test this? Well, we could just put another
name in there and see what happens.
For example, change the name in the DESCRIPTION file to `XMLa'.
First check that there is no package installed with the name
XMLa. In other words, try to load it in R
\begin{verbatim}
library(XMLa)
\end{verbatim}
If you get an error, then you can install this new
version of the package and try to load it in R.
If it does load, then that confirms our hypothesis.

Note that if we didn't confirm that there wasn't already a package
there named `XMLa', our test would not have been conclusive. Instead,
when we installed the new version of the package XMLa, it could have
been installed under some other name and then we would have loaded the
original one.  It is important to have tests that give you proof of
what you are testing and only what you are testing.  If there are
other potential explanations, you have to eliminate them.


Let's suppose that didn't have the XML source package around. Instead,
we copied a DESCRIPTION file from one of the packages that we had
installed previously.  We know that R adds the DESCRIPTION file to the
installation directory of the package when it is actually installed.
So we can look in there for a sample DESCRIPTION file.  We copy that
file into our package skeleton.  Again the phone rings.  We chose not
to answer it as we read about what can go wrong when you get
interrupted in the middle of a task and don't start exactly from where
you left off. That can be a waste of a half hour or more!  So we edit
the DESCRIPTION file.  We go through the fields and change the ones we
understand.  Having done that, we are ready to re-install.
Again, we do
\begin{verbatim}
R CMD INSTALL myPackage
\end{verbatim}
Again, there are no discernible problems.
So we load the package in R and this time all is well.
Now let's take a look at one of the functions
in our package, say myFunction.
When we type its name so that R will simply print it,
R tells us that there is no such variable!
Again, checking our spelling is a good idea.
After this, we might as well see what variables
are defined in this package.
So we use the \SFunc{objects} function:
\begin{verbatim}
R> objects("package:myPackage")
\end{verbatim}
And much to our astonishment, the result
is the empty vector, i.e. no variables
are defined.  

What is the problem.
Again, break down what we did into high-level steps,
and identify which one is causing the problem.
Then break that step into its sub-steps and examine
these. 
Since we had good success in the other situation
from examining the output of the
R CMD INSTALL step, let's do that again.

Again, look through the output and compare this with what we expected.
Pretty quickly we come across the fact that we are installing a binary
package.  Well ours is a source package, so why does it say it is a
binary package.  So let's examine this.  It is hard to figure this out
without reading the documentation or alternatively the code.  I find
it as easy to look at the INSTALL script in $R_HOME/bin.  After
loading that into an editor, we can look through it for occurrences of
the word binary.  It is probably simplest to first look for this word
within comments. This means we don't have to understand
the details of the code to understand why it thinks the package is
binary.
And if we are very lucky, the nice R developers will
have given us some helpful hints in English.
As we scan through this file, we find a comment
\begin{verbatim}
  ## Figure out whether this is a source or binary package.
  ## <NOTE>
  ## If DESCRIPTION has a @samp{Built:} entry, this is a binary package.
  ## This is the right test, but not available for packages installed
  ## prior to R 1.4.  For some time, we thus checked for the existence
\end{verbatim}
So that tells us the problem - our DESCRIPTION file has 
a Built: field. And that makes sense as the INSTALL
process adds one to the DESCRIPTION file when it is installed.
Since we copied the installed version, we have a problem.
Removing the Built: field allows things to work.
Of course, there may (and are) other fields that we also
need to remove for everything to really work.








\section{Have alternatives}


