There are (currently) two chapters related to programming:
\begin{itemize}
\item The first is more of an introduction to the syntax and mechanics
  of programming.
\item The second concerns itself with programming style and how to
  think about composing a program.  This is at a higher-level than
  syntax and ``tricks''.  The material is about how to identify
  components that make up a program and how to consider trade-offs
  between efficiency, extensibility, generality and reusability.  As
  such, it is more language-independent than the programming details
  discussed in the first chapter.
\end{itemize}
The first chapter will discuss different programming language styles,
e.g. functional, procedural, object-oriented, vectorized, visual and
so on.  It will point out the relative merits and limitations of each
of these and try to illustrate when each is a good fit for a problem.
This material is in the same style as the second chapter.  However,
then we move to discussing the particulars and details of different
programming languages commonly used in information technology: C/C++,
Java, Perl, Python, S (R and S-Plus), Matlab (and Octave), Excel.


\chapter{General Programming}
\section{Style}
Meta-information
 \subsection{Variable Names}
 \subsection{Formatting}
   Indentation, etc.
 \subsection{Constants}
 \subsection{Vectorized Operations}
 \subsection{Classes}
 \subsection{Modules}
   Chapters, Modules

 \subsection{Versioning}
   Version information, CVS/Subversion.

 \subsection{Dependencies} 
  Make
 
\section{Efficiency}
 \quote{Premature optimization is the root of all evil}
 \quote{It is hard to debug optimized code than to optimized debugged
 code.}

\begin{itemize}
  \item  Allocation of objects ahead of time.
  \item  Array-length doubling.
\end{itemize}

\section{Portability}

\section{Debugging}
 Approaches
  Tools
  \subsection{Printing - printf}
  \subsection{Trace}
  \subsection{Exceptions}
  \subsection{Type Checking}

  \subsection{Random Bugs}

\section{Testing}
  \subsection{Validation Suites}
    Comprehensive and ``typical''.
  \subsection{Boundary Conditions}
    Extreme cases \\
    Random tests.

\section{Documentation}
   Comments and formal documentation files.
   Synchronization.

\section{Reusing Other People's/Existing Code}
 \subsection{Identifying Software}
 \subsection{Integrating code - InterSystem Interfaces}
   \subsubsection{C}
   \subsubsection{Java}
   \subsubsection{Perl}
   \subsubsection{Python}

\section{Other Languages}

\section{Common Issues}
  More of a cookbook
   \subsection{Idioms}
   \subsection{Problems/Errors}

\section{Case Study}
