
\chapter{Data Ready to Analyze}

A series of examples that introduce I/O,
data manipulation within R, EDA through plotting,
and how to use intuitive statistical methods (clustering, support vector machines,
smoothing splines) in packages/functions.

\section{Rainfall: Colorado front range}

\paragraph{Data}
These data form a list of daily precipitation data from a network of 56 stations 
from the Front Range (Nychka).  List components for FRobs:

\begin{itemize}
\item precip: a list where the kth element is the time series of daily precipitation observations (100th of an inch) for the kth station
\item time: a list where the kth element are times of daily precipitation observations (in years with decimal fraction) for the kth station.
\item     loc: 56X2 Longitude and latitude matrix of the station locations
\item  lat: vector of station latitudes
\item lon: vector of station longitudes
\item  info: a 56X10 matrix of different quantities related to each station
\item Stot: Mean total summer precipitation estimated for each station
\item  elev: Station elevation 
\end{itemize}

\section{Traffic: Flow and Occupancy on California freeways} 

\paragraph{Data}
These data have been collected by loop detectors at one particular
location of eastbound Interstate 80 in Sacramento, California.
There are six columns and 1740 rows in the data set.
The rows correspond to successive five minute intervals from March 14 to 20,
2003, 
where the data values in a row report the flow (number of cars) and occupancy
(the proportion of time there was a car over the loop) in each
of three lanes on the freeway. 
Lane 1 is the leftmost lane, lane 2 is in the center,
and lane 3 is the rightmost.
The original data are from the Freeway Performance Measurement System (PEMS)
website
\begin{verbatim}
 http://pems.eecs.berkeley.edu/Public/
\end{verbatim}

In addition there is incident data taken at the time of 
an accident on a section of the freeway.


\section{Super Nova} 

\paragraph{Data}
Collection of 5000 supernova and 5000 other objects, each with
 19 features transformed (Romano). 
Split into balanced training and test sets.  9000 in training set, 1000 in test set.

\section{NASA satellite}

\paragraph{Data} 
The data are geographic and atmospheric
measurements on a 24 by 24 grid covering Central America.
They are monthly averages from January, 1995 to December 2000 obtained 
from the NASA Langley Research Center Atmospheric Sciences Data Center
\texttt{eosweb.larc.nasa.gov/} 
and available on the American Statistical Association
website.
More details about the data, including descriptions of the variables,
are available at \\
\texttt{www.amstat-online.org/sections/graphics/dataexpo/2006data.php}

\section{Wireless network data from Mannheim}

\paragraph{Data}
The data are collected from 6 wireless base stations within a 
university building in Mannheim (CRAWDAD). 
Person walks through the building and measures the 
signal from a hand-held wireless device to each of the 
base stations. 
Persons orientation is included, measurements of signal strength
are take ad they turn through 360 degrees. 

