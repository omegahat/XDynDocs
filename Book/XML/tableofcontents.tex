\section{Motivation}
What are the problems with dealing with data delivered in ASCII files?
rectangular tables? etc.  Essentially, each user must provide the
meta-information and this is wasteful, error prone and makes it more
difficult to automate.

We want the ability to exchange files  that are easy to transport
e.g. in mail, and are application-independent.
But we also want to be able to supply meta-information
not only about variable names and data types, but also 
things such as documentation about a dataset, missing value
identifiers, author information, complex data types that are
efficiently represented (e.g. longitudinal data,
graphs).  And importantly, we will want
to include additional information in the future
either generally or for particular projects/domains/disciplines.
So we need an extensible format that can handle arbitrary data
and that is 

R's format is fine, so is Matlab's, SAS's etc. but they are tied
currently to a particular piece of software and are not easily
extensible at the representation level but only in the languages.  XDR
is fine in that it is general.  All of these are binary and so harder
to work with directly in mail, etc.

XML is designed to meet most of these needs.  What is XML? It is the
eXtensible Markup Language.  If we think about what we want it is
simply a way to describe the many aspects of data in a general way.
We want to be able to convey a sequence of integers or real values, a
data frame, a complex data structure consisting of several fields of
different types or a complex graph with associations/connections
between nodes, or text documents consisting of chapters, sections,
code, data, etc.  For this to work, we need only be able to identify
the different pieces of the data (e.g. the numbers, the nodes and
edges in the graph, the records in a data frame).  And to do this, we
need a language to identify or markup these different elements within
the data and to describe their characteristics.

XML provides a general markup language.  It is possibly helpful to
think of XML by considering its more specific relation, HTML used to
specify Web pages.  In HTML, we have a fixed set of tags or elements
that we can use to create a page.  These include things like <P> for
paragraph, <H1>, <H2>, etc. for section headers and titles, <table>,
<tr> and <th> for elements of a table and its rows and cells, and then
things like links such as <a href="myPage.html">Click here</a>
which uses attributes of the <a> element to specify the location
to which the browser would jump when the reader clicks the link.

HTML also permits the specification of document information such as
the title that is separate from the body or content.
\begin{verbatim}
<html>
<head>
<title>A Title for the browser's Window</title>
<link style....>
</head>
<body>
</body>
</html>
\end{verbatim}

XML by itself is not very useful.  It is just another attempt to
encourage the use of a standard that can be used for exchanging data.
If anything, it is a constraint as we should use it to increase the
number of people and applications with we can share data
but it is something we have to implement.
And that is where those that have developed XML have helped
greatly. is What makes it powerful is
that there is software in almost all programming languages
to parse, process, create and generally work with XML
documents.

Why is XML really important and not just another standard
that we can optionally use?
The reason is that it is become pervasive.
Web Services, databases, etc.  are 

\section{Basics of XML}
Definition  of terms.
Examples

Considerations
 closing tags
 case sensitive
 namespaces

\section{Parsing}
XSL and specific programming language interfaces.

\subsection{Basic Tree Model - DOM}

\subsection{Hybrid in situ replacement}

\subsection{Efficiency and SAX}

\subsection{Example - Case Study}

\subsubsection{DataSets}
 GGobi, OIM, etc.
PMML.

\subsection{e.g. GML}
Geographic

\subsection{Genetics/Bioinformatics}

\subsubsection{PubMed}
\subsubsection{HTML Documents}

\section{Web Services SOAP}

\section{Meta Programming}
Reading Schema and WSDL files
to create.



\section{Dynamic Content and Documents}
XSL and including output from Statistics.

\section{Creating XML}
Code to generate XML.

\section{Designing an XML DTD/Schema/Document Class}

Relate to schema for RDBMS. Synchronize example with that chapter.