
\section{Introduction}

Intrusion detection systems monitor the use of computers and networks, 
searching for unauthorized use and attempts to deny users access to services.
In order to develop an effective intrusion detection system, 
researchers need a means of evaluating the system and its ability to detect 
intrusions.  
The Information Systems Technology Group at Lincoln Laboratory at MIT 
prepared a body of data for this purpose.  
These data represent the activity over a nine-week period of a 
Local Area Network consisting of hundreds of computers and thousands of 
additional, external computers on the Internet.
For confidentiality reasons, network traffic on a real network could
not be used, and due to the time constraints and unreliabilty in cleaning
real data to remove sensitive information, the Lincoln Labs simulated 
the network activity. 

The simulation was carefully designed to represent real network activity
on a local Air Force base.  (The project was sponsored by the Defense Advanced 
Research Projects Agency and Air Force Research Laboratory.
For several months, the Lab collected 
statistics on the activity of the Air Force base network, including
how people such as administrative assistants, managers, programmers,
and system administrators, performed their day-to-day activities, like 
preparing documents, reading and responding to email, downloading data 
from the Web, compiling and testing programs, and repairing systems.
The simulation generated random activity that matched summary statistical
measures of the real network.  For example, the simulation "wrote" email 
using unclassified documents that matched the word frequencies of the 
real email content.  Over twenty network services were simulated
including mail, web, and file transfer services, and the activity of 
each of these services was 'sniffed' and recorded. 
In addition to simulating normal activity, 32 different types of 
attacks were inserted in the data. These were a mix of older,
current, and novel attacks, some of which were chosen from publicly
documented attacks.
In order to develop techniques for detecting intrusions into networked 
systems, we first must be able to characterize the normal and anomalous
behavior of the system.
The data we will focus on includes all the TCP (Transmission Control Protocol) 
information that identifies which machines and applications were communicating
on the network.  


\section{The Internet and Transmission Control Protocol}
When an application like your Web browser needs to contact
the outside world to retrieve a web page, the communication is handled 
using Transmission Control Protocol (TCP) and Internet Protocol (IP).
These protocols are a set of conventions for using networks to transfer
data on the internet.
For example, to download the page http://www.nytimes.com, the web browser on
your computer connects to the machine that hosts the www.nytimes.com 
Web site, and proceeds directly to the Web site "department''. 
Each department has its own door or \textit{port}.
This is port 80 for HTTP - the HyperText Transfer Protocol - used for
accessing Web pages.  
The browser establishes a 2-way (bi-directional)
connection with the www.nytimes.com machine on this port, and 
it uses this connection to send and receive information.
The browser sends information such as
\begin{verbatim}
 GET /index.html HTTP/1.1
 \end{verbatim}
requesting a particular page for you to browse, and 
the Web server sends the resulting page back along this connection.

But how does the Web browser find the www.nytimes.com machine?
Further, rather than sending entire files across the internet, 
the delivery system breaks the communications into
smaller segments and sends them in separate packets.  
But what if packets get lost on their travels from one application 
to the other, or what if they arrive out of order?  
The protocols provide ways to correct for all of these problems. 

In order to answer the first question, we need to understand the 
topology of the internet. 
The internet is made up of several physical networks that are
connected by routers or gateways, computers that are connected
to multiple networks that shuffle packets from one network to another.
But, to the user, the internet is a single virtual network. 
To find a computer on a network, it must have 
an Internet address, or IP address.
These are unique 32-bit integers assigned by a central authority,
INTERNIC, the Internet Network Information Center.
IP addresses are typically expressed as four 8-bit integers, 
e.g. 128.xx.xxx.xx, but conceptually, the address consists of a 
pair of identifiers, a network ID and a host ID. 
The host ID represents a connection to a network.
If a computer belongs to more than one network then
it must have more than one IP address, and if a computer moves
from one network to another then its IP address must change.

The TCP and IP protocols are very flexible in that almost any 
underlying physical network can be used to transfer data.
The Internet Protocol is an abstraction of the physical, interconnected
networks; it defines a packet delivery system that is
unreliable, best-effort, and connectionless.
That is, it is a packet delivery system becuase the information
to be transferred is divided into small packets, it is
unreliable because delivery of packets in not guaranteed,
best-effort because packets are discared only when network
resources are exhausted or the network fails, and 
connectionless because packets are sent independently
from eahc other and as a result may arrive out of order,
or be lost or duplicated.


The Address Resolution Protocol resolves IP addresses with the physical
address.  The physical address of the machine is tied to the physical
characteristics of the network.  
The ARP forms a network communication system that accepts IP data
and uses the physical address to get it to its destination.



========================

Coceptual Layers
Application, Transfer Service, Connectionless Packet delivery,
Network

Layers of service and protocol

Application program
Application communication
FTP, SMTP, ...

Transfer Service Makes connection
TCP, UDP

Connectionless packet delivery system
IP accept request to send packet

Network Communication - accepts IP data and 
uses physical address to get it there
ARP (address resolution)

Datagram
encapsulation
Fragmentation
Time to live 
Checksum

Routing
Direct delivery
Indirect delivery
Routing table
ICMP

=======================================
Lincoln Labs -- creation of the data

From Duncan's setup.xml


==============================
Our manipulation of the data 

Examples of the code to get data out and plot.
