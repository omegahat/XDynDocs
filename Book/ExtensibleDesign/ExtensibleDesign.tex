\chapter{Programming with Extensibility in Mind}

\textsl{
The content in this chapter could be merged into 
the second chapter on programming and the
chapter on writing software and R packages.
However, it may make sense to keep it separate and not 
confound it with other topics. The ``writing software''
chapter is already overladen with topics.
}

\begin{summary}
  This chapter is about the philosophy of, and practical approaches
  to, writing code that is extensible and reusable.  We outline why
  reusability is good, including avoiding re-testing when we make
  changes and of course the ability to avoid writing new similar code.
  We contrast the trade-offs of over-generalizing and not recognizing 
  generalities.  We try to characterize when such generalities might
  be present and how to recognize them.
  

  The chapter discusses how we write R functions using default values
  so that they allow callers to customize their behavior without
  having to specify the values.  We move on to object oriented
  programming, both S3 and S4 classes and compare these with
  C++/Java-style OOP.  We try to provide some real-life case studies 
  describing how we design classes and make decisions about the class
  hierarchy and method definitions.

  The purpose of this chapter is to encourage the reader to recognize
  that designing with flexiblity and extensibility in mind is
  empirically a good thing as we tend to reuse software.
\end{summary}

\section{The Purpose of Extensible, Reusable Code}

\section{Extensibilty via Default Arguments}

\section{Object Oriented Programming}

\section{S3 Classes \& Methods}

\section{S4 Classes \& Methods}

\section{Integrating S3 \& S4}

\section{C++/Java-style classes \& methods}


