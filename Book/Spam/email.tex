
Electronic mail, usually called e-mail, consists of simple text messages -- 
a piece of text sent to a recipient via the internet. 

\subsection*{E-mail Clients}

To read e-mail, we use an e-mail client, which is software
such as Outlook Express, Eudora, and pine.
If you subscribe to a free e-mail service such as
Hotmail or Yahoo, then to read your e-mail 
you use an e-mail client that appears in a webpage.
With commercial providers such as AOL, you 
can use the provider's e-mail reader or a web-browser. 
All e-mail clients typically provide four basic 
functions: 

\begin{itemize}
\item View a list of the messages in your mailbox.

\item Select a message from the list, and read the 
contents of the message.

\item Create a new message and send it. 

\item Handle attachments -- i.e. add attachments to a message that you 
send and save attachments to a message you receive. 
\end{itemize}


\subsection*{E-mail Servers}
To send and receive e-mail you need a special 
computer on the Internet for the client to connect
to and hand over the e-mail message for delivery.
This computer runs a software application to
provide e-mail service, and so is called an
e-mail server.

There are millions of computers on the Internet
running software applications that act as 
Web servers, FTP servers, telnet servers, 
e-mail servers, etc. 
These applications run continuously on the server
machine and they listen to specific ports, 
waiting for people or programs to attach to the port. 
The SMTP (Simple Mail Transfer Protocol) server
handles outgoing e-mail.
This server listens on port number 25 for 
anyone who wants to send a piece of e-mail.
Your e-mail client interacts with the SMTP
server to send mail.

The following is an example of how your STAT 133 instructor
might send an e-mail to you. might

\begin{enumerate}
\item The instructor logs on to the Statistical Computing Facility
computers and opens an e-mail client, say pine. 
She composes a message in the e-mail client to you that says:

 Dear Student, \\ 
   You received an A for your project.\\
 Sincerely,\\
 Your instrcutor 

\item The instructor hits the send button, Ctrl-X in pine, to send the message
to the student with email address {\texttt student@hotmail.com}.

\item Pine, the e-mail client, connects to the SMTP server at 
stat.berkeley.edu using port 25.

\item Pine gives the SMTP server the address of the
sender, the address of the recipient, 
and the body of the message.

\item The SMTP server takes the "to" address 
(student\@hotmail.com) and breaks it into two parts:

\begin{enumerate}
\item The recipient name (student)
\item The domain name (hotmail.com) 
\end{enumerate}

Note that if the instructor had sent the e-mail to the
student's class account, eg. {\texttt s133cs@stat.berkeley.edu},
then the SMTP server would simply hand the message 
to the server that handles incoming mail at 
stat.berkeley.edu.  The incoming mail is handled by an IMAP server 
(more on it later).

\item Since the recipient is at another domain, 
SMTP needs to communicate with that domain.
The SMTP server has a conversation with a 
Domain Name Server, or DNS, to find the IP of the 
SMTP server for hotmail.com. 
The DNS replies with the one or more IP addresses 
for the SMTP server(s) that Hotmail operates.

\item
The SMTP server at stat.berkeley.edu
connects with the SMTP server at Hotmail using port 25. 
It has a very simple text conversation with the other SMTP,
and gives the message to it. 
The Hotmail server recognizes that the domain name 
for student is at Hotmail, 
so it hands the message to Hotmail's IMAP server, 
which puts the message in student's mailbox. 
\end{enumerate}

What does this conversation between the two SMTP servers look like?
If is specified in public documents
called Requests For Comments (RFC).
The SMTP is defined in RFC 821 and RFC 1123.
It is very simple; it demands an ordered data stream of 
7-bit US-ASCII characters.
The conversation is initiated by the sender issuing SMTP commands to
the receiver. 
The receiver replies to the sender with numeric reply codes, 
followed by a text string with additional information about the reply code.

The SMTP server understands very simple text commands like HELO, MAIL, 
RCPT and DATA. The most common commands are:

\begin{itemize}
\item HELO - introduce yourself
\item EHLO - introduce yourself and request extended mode
\item MAIL FROM: - specify the sender
\item RCPT TO: - specify the recipient
\item DATA - specify the body of the message 
\item RSET - reset
\item QUIT - quit the session
\end{itemize}

The following are some reply codes sent by the receiver SMTP server.

\begin{itemize}
\item 211  	System Status or system help reply
\item 220 	domain Service ready
\item 221 	domain Service closing transmission channel
\item 250 	Requested action OK and completed
\item 354 	Start mail input; end with .
\item 421 	Domain service not available, closing connection
\item 450 	Mailbox unavailable, requested mail action not taken
\end{itemize}

In a typical exchange, after the sender contacts the receiver,
the receiver replies with code 220 indicating that it is ready. 
The sender then sends the HELO command with the client host as an argument. 
The HELO command identifies the sender to the receiver, 
and the receiver will respond with a reply code 250. 
This tells the sender that the connection is open and ready to go. 
The next step in the transaction identifies and confirms host addresses 
for both the sender and receiver servers.
After introducing itself, the e-mail client 
indicates the "from" and "to" addresses, and then 
issues the command DATA to asks the receiver if it is ready to receive the 
message.
The receiver replies with a 354 code to indicate that the 
sender can deliver the message.
This transmission ends with a lone period `.' on a line. 
After sending the message, the sender the issues the command QUIT,
and the receiver responds with a 221 reply code.
You can, in fact, telnet to a mail server machine at port 25 and have 
one of these dialogs yourself -- this is how people "spoof" e-mail.


\subsection*{The Anatomy of an E-mail message}

An e-mail consists of two parts, the header and the body.
The header contains information about the e-mail such
the sender's address, the recipients address, and the date of 
transmission.
This information is relayed in a special format 
of KEY: VALUE pairs that conform to RFC 822.
Below is a sample header from a message found in the
SpamAssassin website.

\begin{verbatim}
Return-Path: whisper@oz.net
Delivery-Date: Fri Sep  6 20:53:36 2002
From: whisper@oz.net (David LeBlanc)
Date: Fri, 6 Sep 2002 12:53:36 -0700
Subject: [Spambayes] Deployment
In-Reply-To: <LNBBLJKPBEHFEDALKOLCIEJABCAB.tim.one@comcast.net>
Message-ID: <GCEDKONBLEFPPADDJCOECEHJENAA.whisper@oz.net>
\end{verbatim}
                                                                               
Notice the keys are Return-Path, Delivery-Date, From, Date,
Subject, In-Reply-to, and Message-ID. 
The value follows the keyword.
For example, in the above header, the value of the From key 
is whisper\@oz.net (David LeBlanc).

Some of these keys are mandatory such as Date, From, and To (or 
In-Reply-To, or BCC).
Others keys are optional but widely used, such as Subject, Cc, 
Received, and Message-ID.
Many keys are ignored by the mail system, but the entire header 
is relayed on to the recipient's server whether or not it is 
recognized. 
for example, headers starting with 'X-' are for personal application 
or institution use.

The "Received" header lines are important because they allow
the message to be tracked. As a message makes its way to the
intended resipient, servers add additional "Received" lines
to the header. 
These lines are added to the top of the header so  the 
first server to receive the message will be at the bottom
of the list of Received lines.
Note that spammers often add fake Received lines to their
e-mail headers.

Below are some typical header keys:

\begin{itemize}
\item Message-Id - a unique identifier for the e-mail, assigned by the 
originating server 
\item Return-Path - specifies the ENVELOPE sender's address; 
         BOUNCED mail gets sent to this address.
\item Date - added by the e-mail client
\item CC - the recipients of a ``carbon copied" email
\item Reply-To - the address set by the sender to which the 
recipient can reply 
\item MIME-Version - used for encoding binary content as attachments.
\item X-header key - for personal application or institution use. 
\end{itemize}

A value may be continued on a second line of the header, in which
case the line will be indented and begin with a tab character or
blank spaces.

\subsection*{Attachments}

The body of the e-mail is separated from the header by a single
blank line. 
When an attachment is added to an e-mail message, the attachment
is included in the body of the message. 
Even with attachments, e-mail messages are still only
text messages. 

An Internet standard called MIME, 
Multipurpose Internet Mail Extensions, 
specifies how messages may be formatted and how to separate 
the attachments from the message. 
Information about the MIME encoding is provided through header fields, 
which are specified in an RFC. 

The Content-Type key is used to describe the content of a component
or of the entire body. 
The value provides the top-level type and subtype using the 
the syntax:  top-level/subtype; parameter.
Parameters may be required or optional.
Below is an example of a content-type where the 
top-level is ``multipart" which indicates there will be several 
documents in the body of the message, and the ``mixed" subtype  tells
us that each document may be of a different type.

\begin{verbatim}
Content-Type: multipart/mixed;
      boundary="----=_NextPart_000_00DE_01511A02.DB1A02A0"
\end{verbatim}

In this example, the Content-Type field tells the receiving e-mail program 
that this message has more than one component, and each component will be 
separated by the string of characters 
\begin{verbatim}"----=_NextPart_000_00DE_01511A02.DB1A02A0"\end{verbatim}.
The boundary string marks the beginning of each component.
It is prefaced with two hyphens in all instances.
The boundary string is also used to denote the end of the message, 
where it is both prefaced by two hyphens and followed immediately by two 
hyphens. 
The receiving email program knows when the last component of the message
has been read when it reads the boundary string followed by two hyphens.

Each component of an e-mail must be prefaced by this 
boundary string, optional MIME information, and a mandatory blank line. 
If the blank line is missing, the recipient's email program may have 
difficulty telling where the header information stops and the text of the 
message begins.

There are seven top-level types: text, image, audio, video, application,
multipart, and message.  Other examples of Content-Type follow:

\begin{verbatim}
Content-type: text/html; charset=euc-kr;

Content-Type: application/zip; name="testFile.zip"
\end{verbatim}

The first indicates that the message will be in HTML format using a
Korean character set.  The second indicates that the component will
is a zip file, and it sender indicates that it should be saved as
``testFile.zip".
Binary files (such as a compressed archive) can be
sent as attachments. 
In such cases, the sender must first encode the binary
file so that it can be sent over the Internet. 
One common encoding scheme is known as base64. 

We conclude with two sample e-mail messages.
The first is a plain text e-mail with no
attachments.  It consists of an instructor's response
to an e-mail inquiry sent by a student. 

\begin{verbatim}
 
From nolan@stat.Berkeley.EDU Mon Feb  2 22:16:19 2004 -0800
Date: Mon, 2 Feb 2004 22:16:19 -0800 (PST)
From: nolan@stat.Berkeley.EDU
X-X-Sender: nolan@kestrel.Berkeley.EDU
To: Txxxx Uxxx <txxxx@uclink.berkeley.edu>
Subject: Re: prof: did you receive my hw?
In-Reply-To: <web-569552@calmail-st.berkeley.edu>
Message-ID: <Pine.SOL.4.50.0402022216120.2296-100000@kestrel.Berkeley.EDU>
References: <web-569552@calmail-st.berkeley.edu>
MIME-Version: 1.0
Content-Type: TEXT/PLAIN; charset=US-ASCII
Status: O
X-Status:
X-Keywords:
X-UID: 9079
  
Yes it was received.
 
-------------------------------------
 
On Mon, 2 Feb 2004, txxxx wrote:
 
> hey prof .nolan,
>
> i sent out my hw on sunday night. i just wonder did you receive it.
> because i am kinda scared thatyou didnt' receive it.
> like i just wonder how do i know if you got it or not, since the cal
> mail system is kinda weird sometimes.  thanks
>
> txxxx
>
\end{verbatim}    

The second e-mail consists of an message and two attachments from
a student to the instructor that has been forwarded on by the 
instructor to the teaching assistant.
The three periods at the end of each attachment indicates
that the attachment continues quite a bit longer.
the first attachment is a pdf file and the second is an HTML file.
The forwarded message is a plain text file.

\begin{verbatim}
From nolan@stat.Berkeley.EDU Mon Feb  2 22:18:56 2004 -0800
Date: Mon, 2 Feb 2004 22:18:55 -0800 (PST)
From: nolan@stat.Berkeley.EDU
X-X-Sender: nolan@kestrel.Berkeley.EDU
To: Gang Liang <liang@stat.Berkeley.EDU>
Subject: Assignment 1 sorry (fwd)
Message-ID: <Pine.SOL.4.50.0402022218470.2296-201000@kestrel.Berkeley.EDU>
MIME-Version: 1.0
Content-Type: MULTIPART/Mixed; BOUNDARY="_===669732====calmail-me.berkeley.edu===_"
Content-ID: <Pine.SOL.4.50.0402022218471.2296@kestrel.Berkeley.EDU>
Status: RO
X-Status:
X-Keywords:
X-UID: 9080
 
--_===669732====calmail-me.berkeley.edu===_
Content-Type: TEXT/PLAIN; CHARSET=US-ASCII; FORMAT=flowed
 
Content-ID: <Pine.SOL.4.50.0402022218472.2296@kestrel.Berkeley.EDU>
 
 
   
---------- Forwarded message ----------
Date: Mon, 02 Feb 2004 21:50:47 -0800
From: Yyyy Zzz <Zzz@uclink.berkeley.edu>
To: nolan@stat.Berkeley.EDU
Subject: Assignment 1 sorry
 
I am sorry to send this email again, but my outbox told me that 
the last email only send 1 attached file. 
I am send ing this again to make sure you recieve the all the necessary files.
You and sorry for the inconvenience.
  
--_===669732====calmail-me.berkeley.edu===_
Content-Type: APPLICATION/PDF; CHARSET=US-ASCII
Content-Transfer-Encoding: BASE64
Content-ID: <Pine.SOL.4.50.0402022218473.2296@kestrel.Berkeley.EDU>
Content-Description:
Content-Disposition: ATTACHMENT; FILENAME="PLOTS.pdf"
 
JVBERi0xLjEKJYHigeOBz4HTDQoxIDAgb2JqCjw8Ci9DcmVhdGlvbkRhdGUgKEQ6MjAwNDAy
MDIxMTIwMTEpCi9Nb2REYXRlIChEOjIwMDQwMjAyMTEyMDExKQovVGl0bGUgKFIgR3JhcGhp
Y3MgT3V0cHV0KQovUHJvZHVjZXIgKFIgMS44LjEpCi9DcmVhdG9yIChSKQo+PgplbmRvYmoK
MiAwIG9iago8PAovVHlwZSAvQ2F0YWxvZwovUGFnZXMgMyAwIFIKPj4KZW5kb2JqCjQgMCBv
YmoKPDwKL1Byb2NTZXQgWy9QREYgL1RleHRdCi9Gb250IDw8IC9GMSA2IDAgUiAvRjIgNyAw
IFIgL0YzIDggMCBSIC9GNCA5IDAgUiAvRjUgMTAgMCBSIC9GNiAxMSAwIFIgPj4KPj4KZW5k
b2JqCjUgMCBvYmoKPDwKL1R5cGUgL0VuY29kaW5nCi9CYXNlRW5jb2RpbmcgL1dpbkFuc2lF
...
 
--_===669732====calmail-me.berkeley.edu===_
Content-Type: TEXT/HTML; CHARSET=US-ASCII
Content-Transfer-Encoding: BASE64
Content-ID: <Pine.SOL.4.50.0402022218474.2296@kestrel.Berkeley.EDU>
Content-Description:
Content-Disposition: ATTACHMENT; FILENAME="Stat133HW1.htm"
  
PGh0bWwgeG1sbnM6bz0idXJuOnNjaGVtYXMtbWljcm9zb2Z0LWNvbTpvZmZpY2U6b2ZmaWNl^M
PGh0bWwgeG1sbnM6bz0idXJuOnNjaGVtYXMtbWljcm9zb2Z0LWNvbTpvZmZpY2U6b2ZmaWNl^M
Ig0KeG1sbnM6dz0idXJuOnNjaGVtYXMtbWljcm9zb2Z0LWNvbTpvZmZpY2U6d29yZCINCnht^M
bG5zPSJodHRwOi8vd3d3LnczLm9yZy9UUi9SRUMtaHRtbDQwIj4NCg0KPGhlYWQ+DQo8bWV0^M
YSBodHRwLWVxdWl2PUNvbnRlbnQtVHlwZSBjb250ZW50PSJ0ZXh0L2h0bWw7IGNoYXJzZXQ9^M
d2luZG93cy0xMjUyIj4NCjxtZXRhIG5hbWU9UHJvZ0lkIGNvbnRlbnQ9V29yZC5Eb2N1bWVu^M
...

--_===669732====calmail-me.berkeley.edu===_--
    
\end{verbatim}

