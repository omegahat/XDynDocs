
The directory name can be relative to where you are or absolute. 
Suppose you are in the directory /home/nolan/stat133, and it contains a directory notes. 
The next three commands all change the directory to /home/nolan/stat133/notes.

\begin{itemize}
\item \shellCmd{cd notes} -- The directory name is relative to the current directory
\item \shellCmd{cd /home/nolan/stat133/notes} -- Here we have supplied
the full pathname to find the directory

\item \shellCmd{cd \~/stat133/notes} -- We use the symbol \shellCmd{\~} to 
 refer to the user's root directory, which in this example is /home/nolan.

\item \shellCmd{cd ../} -- This relative change is to the directory above the current directory, i.e. to /home/nolan/stat133

\item \shellCmd{cd ../stat205} -- This relative change is to the directory 
at the same level of the tree as the current directory i.e. /home/nolan/stat205 


\item \shellCmd{tail \textit{filename}} -- shows the last lines of a file
\end{itemize}

{Example}
Let's create a system of files and directories that matches the
handout from Friday.

\begin{verbatim}
> mkdir Test
> cd Test
Test> touch x
Test> touch file1.tex
Test> mkdir A
Test> mkdir A/B
Test> cd A
Test/A> touch file2.tex
Test/A> touch file3.csv
Test/A> mkdir C
Test/A> touch C/x
Test/A> touch B/file4.doc
Test/A> touch B/file5.txt
Test/A> mkdir B/D
Test/A> touch B/D/x
Test/A> touch B/D/x2
\end{verbatim}


From within the Test directory that we set up 
we see the directory A and the two files x and file1.tex.
\begin{verbatim}
> ls
A  myfile.tex  x
\end{verbatim}

The \shellKey{-R} option say recurse your way through the tree of subdirectories excuting the \shellCmd{ls} command as you go.

{\small{
\begin{verbatim}
> ls -R
.:
A  file1.tex  x
./A:
B  C  file2.tex  file3.csv
./A/B:
D  file4.doc  file5.txt
./A/B/D:
x  x2
./A/C:
x
\end{verbatim}
}}


{\shellArg{*} -- the wild card}

The * symbol matches any number of characters (except the /). It can be quite handy when looking for files that have particular type.

Below we list only those files with the filetype extension of \texttt{tex}
in the subdirectory A:
\begin{verbatim}
> ls A/*.tex

A/report.tex

> ls -R *.tex

ls: No match.
\end{verbatim}

Can you explain why the second command does not find any files with
the filetype \texttt{tex}?

{Piping \shellArg{$|$}}

We can construct more complex shell commands by piping or sending the output 
from one comand to the input of another.
Below we pipe the output from the \shellArg{ls} command to the 
\shellCmd{wc} comand.

\begin{verbatim}
> ls -R | wc

  33      29     140
\end{verbatim}

\shellCmd{wc} -- short for word count, returns the number of newlines, words, 
and bytes in the input file.
Does this mean that there are 33 files in the Example directory?

{\shellCmd{find} --  search for files in a directory hierarchy}

The \shellCmd{find} command may be better suited to the previous task.
\begin{verbatim}
> find -name '*.tex'

./A/report.tex
\end{verbatim}

We can also do fancier finds, such as all tex files 
in my home area that have been modified within the past 21 days.
\begin{verbatim}
> find /home/nolan/ -mtime +21 -name '*.tex' | wc -l

2421
\end{verbatim}

Or, we can find all those files that do not end with .tex
in the working directory.
\begin{verbatim}
> find . -type f -not -name '*.tex'
\end{verbatim}


{\shellCmd{grep} -- print lines matching a pattern}

With \shellCmd{grep} we can search for patterns within a file.
The syntax is\\
\shellCmd{grep [options] PATTERN [FILE...]}.

Suppose we want to find all files that have the words "data frame"
in them.
\begin{verbatim}
> grep -lR 'data frame' *

DataTypes/Assignment.tex
DataTypes/Lists.tex
Introduction/introduction.tex
ShellCmds/.ShellCmds.tex.swp
Traffic/traffic.tex
schedule
\end{verbatim}


To invert the match we can use the \shellKey{-v} option
to the \shellKey{grep} command. 

\begin{verbatim}
> grep -lRv 'data frame' * | more
DataTypes/DataTypes.tex
DataTypes/htmlPDF.pdf
DataTypes/DataTypes.log
DataTypes/DataTypes.aux
DataTypes/DataTypes.pdf
DataTypes/pdfFiles.tex
...
DataTypes/ExampleLatex2.log
DataTypes/ExampleLatex2.aux
DataTypes/ExampleLatex2.pdf
--More--
\end{verbatim}

Here the \shellCmd{more} command formats the output being sent to the 
console so that you see one page at a time, rather than have the 
whole output go whizzing past.
\\
To proceed to the next page of output hit the space bar.
\\
To stop displaying the output hit \shellCmd{q} 

{Other useful aspects of the Shell commands}

\begin{itemize}
\item \shellArg{$>$ and $>>$} -- redirection

At times we want to save the output from a command to a file.
We can do this by redirecting the output to a file. We use the 
single $>$ to direct the output to a new file (it will also overwrite
an existing file), and we use the double $>>$ if we want to add
the output to an existing file.

\item \shellCmd{chmod} -- change permissions on files and directories.

In Unix we can define groups of users, such as all student accounts for 
Stat133, or all faculty accounts, or all graduate studnet accounts. 
The \shellCmd{chmod} command allows us to set specific permissions for
the owner of the file, the group that the owner belongs to, and for
all other accounts.
The \shellKey{-l} option on the \shellCmd{ls} command provides the permissions.

\begin{verbatim}
Test> ls -l
total 4
drwxr-xr-x    4 nolan    nolan        4096 Sep 19 11:54 A
-rw-r--r--    1 nolan    nolan           0 Sep 19 11:53 file1.tex
-rw-r--r--    1 nolan    nolan           0 Sep 19 11:53 x

Test> chmod g+w *
Test> ls -l
total 4
drwxrwxr-x    4 nolan    nolan        4096 Sep 19 11:54 A
-rw-rw-r--    1 nolan    nolan           0 Sep 19 11:53 file1.tex
-rw-rw-r--    1 nolan    nolan           0 Sep 19 11:53 x
\end{verbatim}

\item \shellCmd{sort} -- sort lines of text files
\item \shellCmd{uniq} -- remove duplicate lines from a sorted file
\item \shellCmd{cat} -- concatenate files and print on the standard output
\item \shellCmd{tail} - output the last part of files
\end{itemize}

Process controls
\begin{itemize}
\item \shellCmd{ps} - report a snapshot of the current processes.
\item \shellCmd{kill} - send a signal to a process
\end{itemize}

More advanced shell techniques
\begin{itemize}
\item \shellCmd{sed} --a stream editor.  
A stream editor is used to perform basic text transformations on an 
input stream (a file or input from  a  pipeline).

\item It is possible to write shell scripts, i.e. programs with
loops and variables, and excute these at the command line
\end{itemize}
