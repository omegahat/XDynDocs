
\section{Working with Files in R}

There are two basic R functions that allow us to get information about
files and directories within the file system.  These are
\Sfunction{list.files} and \Sfunction{file.info}.  Using these, we can
do most of the same computations that we can do in the shell with
\executable{ls} and \executable{find}.
The function
\Sfunction{list.files} is used
to find the names of the files within
a folder or recursively through all of the sub-folders
that match a particular pattern, or just all of the file
names.
This includes both files and directories/folders.
We can use this to work with files that have a particular
pattern to their name in the same way that we can use
the shell ``globbing'' to identify files of interest there.
Unlike the shell globbin, this uses a regular expression
as the pattern matching mechanism.

We will first work with the NASA data.
All the files are in the same directory,
say \dir{NASA/Files}.
We can find the names of all the files in the directory
with the R command
<r:code>
fileNames = list.files("NASA/Files")
</r:code>

For interactive use, this is fine.
However, if this is code that we might want to run on Windows, the 
use of / in the directory path could cause problems.
Instead, we could use
<r:code>
fileNames = list.files(paste("NASA", "Files", sep = .Platform$file.sep))
</r:code>
and this will be correct for the system on which it is run.
Regardless of which way we approach this,
<r:var>fileNames</r:var> should be
a character vector giving the names of the files in
that directory.

There are two aspects to consider.
Firstly, there may be some hidden files, e.g
those starting with a '.' which are not displayed
in the directory listing by default.
We can include these in are results using
the parameter <r:arg>all.files</r:arg> and passing it the
value <s:true/>.

The other aspect is that the results only give the names of the 
files and not their paths. So we end up with values
such as
"cloudhigh1.txt"  and "cloudhigh10.txt" 
and not
"NASA/Files/cloudhigh1.txt".
In many cases, this is exactly what we want.
The files may be in the current working
directory and so directly accessible to us
via their simple, unqualified names.
Or we may just want to work with the names
and not the paths for various reasons, 
e.g. to count the number of files.
If, however, we do want the fully qualified names of the
files, we can have these included in the results using the parameter
<r:arg>full.names</r:arg> and again passing the value
<r:true/>.

When we we look at the NASA data, we see that there
are 506 files.
We can compute their extensions using regular expressions
<r:code>
unique(gsub(".*\\.([^.]+)$", "\\1", fileNames) )
</r:code>
And with this, we see we have two "types" of files:
txt and dat files.
How many are there of each?
<r:code>
table(gsub(".*\\.([^.]+)$", "\\1", fileNames) )
<r:output>
dat txt 
  2 504 
</r:output>
</r:code>
So we see that the bulk of the data are in the .txt files,
and we must be careful to treat the .dat files separately.

We can find the names of these .dat files
by specifying a pattern for the files of interest
in a call to  <r:func>list.files</r:func>.
<r:code>
list.files("NASA/Files", pattern = "\\.dat$")
<r:output>
[1] "elevation.dat" "intlvtn.dat"  
</r:output>
</r:code>

The values in elevation.dat are for the entire
grid and not our 24x24 subset.
So we will ignore those
and we will read the values from
intlvtn.dat, standing for "internal elevations".
This can be done easily with
<r:func>read.table</r:func>
<r:code>
elevations = read.table("NASA/Files/intlvtn.dat")
</r:code>
These do not have identifiers for the latitude and longitude for
the locations/cells.

Now we turn to the .txt files.
Again, we need to list just those files.
But having done so, we need to find
the different variables and identify the patterns in the file names.
This is quite easy here, and generally as the files are usually organized
by a human and given names according to some rule.
In this case, each file has the pattern
variable name followed by a number between 1 and 72 and
then the extension .txt.
We need to verify that this is true,
and this illustrates the way that we write code for a particular dataset
empirically rather than from some external, formal specification of the possible contents.




%%% Local Variables: %%% mode: latex %%% TeX-master: t %%% End:
