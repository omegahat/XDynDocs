\subsection*{Preface} 
\addcontentsline{toc}{subsection}{Preface}

The aim of this book is to provide a text for both upper-division
undergraduates and graduate students in the area of statistical and
scientific computing with data.  The emphasis is on both the
high-level concepts and practical idioms of programming languages for
scientific computing, and both data and Web technologies, and
visualization that are increasingly important in the practice of
science.  We are in a remarkable era where data for so many
interesting and important problems are immediately available via the
Web to anyone who is interested.  To make use of it, however, we need
to have a modern vocabulary and set of skills to be able to access and
manipulate such data.  These skills are novel for statisticians as
they are rarely part of the curricula for either undergraduates or
graduates.  For the good of our field and science generally, however,
this has to change. We hope this book will prove to be useful in this
modernization of both the statistics curricula and also the way and
spirit at which statisticians view their discipline as transforming
into data and information science.


Originally, this book was aimed at upper-division undergraduates.  The
intent was to teach them the basics of statistical computing so that
they would be able to access, explore and present real data in the
context of meaningful investigations.  We wanted them to be able to be
productive and engaged in analyzing data and exploring statistical
methodology via simulation.  We hoped that the material would both
provide a strong foundation in being able to work with data, but also
teach them how to learn about new technologies in this fast changing
environment.

We subsequently decided to grow the book to contain additional, less
fundamental but important topics for more advanced students. The aim
was to address Master's students interested in more advanced
computational topics. This subsequently grew into additional topics
that are of interest to PhD students also and on into new technologies that
we envisage becoming increasingly relevant and popular for
statisticians (e.g. visualization with Google Earth, Web based
interactive graphics and composition, inter-system interfaces).  Some
of this material is recent research but very relevant and practical
for the practice of statistics.

We believe that the book will also be of interest to students and
practitioners in other fields.  We cover many fundamental tools of
scientific computing and programming generally.  These include core
tools that are hidden from casual users but which are important to
understand generally as one becomes more sophisticated.  These include
advanced topics such as the UNIX shell, make, linking and loading
compiled code, etc., but also new paradigms for data visualization
using Web technologies and data acquisition.

There are several books that cover many of the topics we
cover. However, there are very few books that cover all, or even a
significant subset of, the topics we present. Secondly, many books
that cover these topics act as relatively brief introductions or
references.  In this book, we have strived to present not only the
technical details and the how-to of the technologies, but also discuss
how to think about them, when to use them, compare different idioms,
paradigms and approaches.  Generally, we try to engage the reader in
thinking about the material and why it was designed that way, what
alternatives might be, and to be able to think about it rather than
simply mimic and connect lines of code.  The hope is that the reader
will have a deeper understanding and an ability to reason about
problems and approaches because of this understanding.  This is an
ambitious goal and may not be the aim of all readers.  However, we do
think that it is important for more statisticians and scientists
generally to become much more expressive computationally and this
requires not just learning the vocabulary, but style and presentation.
Other books very understandbly focus on 

This book is intended to be a textbook but also a guide for
practitioners.  We do not expect that an instructor would teach more
than half of the material in any course.  We do think however it is
useful to illustrate to students that there is more to scientific and
statistical computing than any of basic programming, simple
visualization, data technologies, Web services, advanced computing
techniques, algorithms and computational statistical methods.  We
expect that a curious student might read this book by herself to learn
important foundational material that we would like our good students
to know so that they can be complete statisticians who can exploit the
power of computing to do good data analysis and statistical research.





